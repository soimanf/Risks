\documentclass{article}


% here all the packages
\usepackage[a4paper, total={6in, 10in}]{geometry}
\usepackage{apacite}
\usepackage[round,sort,colon]{natbib}
\usepackage{booktabs,array, graphicx}
\usepackage{authblk}
\usepackage{multicol, multirow}
\usepackage[table,xcdraw]{xcolor}
\usepackage{setspace}
\usepackage{tabulary,tabularx}
\usepackage[table,xcdraw]{xcolor}
\usepackage{appendix}
\usepackage{amsmath}
\usepackage{url, lscape, geometry, rotating}
\usepackage[utf8]{inputenc}
\usepackage{pdfpages}
\usepackage[skip=0.5ex]{subcaption}
\usepackage[singlelinecheck=false]{caption}

\usepackage{rotating} % Rotating table

\usepackage{color,svgcolor}
\usepackage[plainpages=true,verbose]{hyperref}
\makeatletter
\hypersetup{
breaklinks=true,
colorlinks=true,
 linkcolor=blue,
 citecolor=darkgreen,
 urlcolor=darkred,
}
\makeatother

\author[1]{Jean-Guillaume Dumas}
\author[2]{Sonia Jimenez-Garces}
\author[1,2]{Florentina Șoiman}
\affil[1]{Univ. Grenoble Alpes, CNRS, LJK, F-38040 Grenoble, France}
\affil[2]{Univ. Grenoble Alpes, Grenoble INP, CERAG, 38000 Grenoble France}
\affil[ ]{\textit{[Firstname.Lastname]@univ-grenoble-alpes.fr}}

\title{\boldmath Blockchain technology and the crypto-market: A literature survey}

\onehalfspacing
\begin{document}
\maketitle
\author{}

\begin{abstract}
This paper provides a literature survey on the vulnerabilities and
risks of Blockchain technology and the crypto-market. Since their
creation, the crypto-market and Blockchain technology are still very
much challenged and far from the mainstream adoption. We thus here
propose a detailed literature survey focusing on the relationship
between technological characteristics and financial
risks. Furthermore, to complete this study, we propose ways to
determine the likelihood of technological vulnerabilities triggering
financial risks. Additionally, we find a significant relationship
between Blockchain attacks and cryptocurrency volatility, hence
illustrating the relationship between technological vulnerabilities
and financial risk. While we take into consideration the existent
literature on the vulnerabilities and risks associated with the
crypto-market, we cannot ignore the limited work with respect to
crypto-fundamentals. Alleviating this gap will require more
theoretical models and empirical research that encompass the
technological characteristics of this market. Our results are
twofold. First, we show an identified continuity between the
technological risks and financial ones. Secondly, we find that price
stability is disturbed by technological vulnerabilities.
\\
\noindent \textbf{Keywords: Blockchain, Risk assessment, Financial risks, Attacks, Literature survey }
\\
\noindent \textbf{JEL Codes: G10, G15, G19}
\end{abstract}

\pagebreak
\section{Introduction}

Everyone has heard about the enormous potential of the Blockchain
technology and the fact that it might revolutionize business models
and reinvent the contemporary firms and economies. At the same time,
we know that it is still far from keeping all its promises and before
that happens, Blockchain has to first overcome its technological,
organizational and social barriers \citep{Iansiti2017,
  Charles2019}. As professor Karim Beddiar said, “The Internet has
democratized the information, the Blockchain will democratize the
transaction” \footnote{Own translation from original: ” Internet a
  démocratisé l’information, la Blockchain va démocratiser la
  transaction” \citep{Charles2019}}, however, there is still a lot of
work left for research before that happens and a lot of experience to
gain before the technology will mature \citep{Charles2019}. This
global distributed, open and transparent database, which stores and
transfers information of any kind (money, art, science, titles, votes,
etc.) has the potential to create new foundations for the economy and
business sector. Blockchain might be a complex technology, but the
concept behind it is very simple \citep{Tapscott2016, Iansiti2017}.


Inspired from the existing systems and technologies, the solutions
promised by Blockchain seem to be far beyond what we have already
seen. Little by little, Blockchain is taking over many sectors of the
economy and a growing number of organizations are declaring their
enthusiasm and interest in using it \citep{Collomb2016}.Given the
spread of Blockchain-based solutions across various industries and the
growing interest in using it, there is an urgent need for researchers
and market participants to gain understanding of what it means to be
part of the crypto-market.


As previously mentioned, Blockchain needs to overcome a series of
challenges, before becoming a mainstream technology
\citep{Wachsman2019}. According to \cite{Iansiti2017},there are two
dimensions affecting the way technology evolves. The first dimension
represents novelty, referring to the degree of originality and
uniqueness in comparison with the existing systems. This dimension
implies as well the difficulty in seeing the use and innovation of
technology. The second dimension refers to complexity, implying the
extent to which this technology touches various fields, regardless the
market or area of expertise \citep{Iansiti2017, Notheisen2019}. The
same idea is sustained by the results obtained in the surveys
conducted by Deloitte and Underscore companies. While assessing the
Blockchain adoption, they found out that some of the main barriers
are: the technological complexity, regulatory issues, lack of in-house
skills and understanding, security threats and the uncertain
profitability \citep{Pawczuk2019, UnderscoreVC2018}. In 2018,
\cite{Gazali2018} explored the relationship between human conduct and
the intention to invest in the crypto-market. Consequently, they found
out that the attitude towards the crypto-market, the social norms
\footnote{decisions are made based on the actual trends and influenced
  by a mentality such like: “if I lose, at least I am not alone”)},
the risk tolerance and the perceived benefits coming from using this
technology, represent some of the main factors influencing the
interested parties to invest or be part of the crypto-market.


Regardless the high potential and great innovative solutions brought by
Blockchain, this technology gained most of its fame thanks to its
vulnerabilities. The cryptocurrencies’ volatility and the numerous
cyber-attacks suffered by this technology, represent the main driving
factors towards the Blockchain’s popularity. Among the existing
research literature, several studies have addressed the crypto-market
risks. Some with the aim to find solutions to these vulnerabilities
\citep{Bonneau2015, Stewart2018, Ma2018, Goffard2019, Morganti2019,
  Drljevic2019, Patel2020}, while others just to increase general
awareness \citep{Saad2019, Canh2019, Lemieux2016, Gazali2018,
  Lu2019a}.


Consequently, following the review of the existing literature, we
claim that there is a need for further research on risks and
vulnerabilities of the crypto-market. Compared to previous papers,
where risks were usually treated based on their nature (i.e. economic,
political, regulatory, etc.), we intend to provide a parallel analysis
of both the financial and technological risks. We show that these
risks, regardless their nature, have many characteristics in
common. Moreover, we offer ways to determine the likelihood that
technological risks could transform into financial ones and provide a
short empirical demonstration.


This study is a literature-based research. Compared to other areas,
finance is mostly dominated by quantitative type of analysis. However,
we believe that a new field of research like the crypto-market, would
greatly benefit from such a powerful scrutinizing tool that explores
the existing papers and informs the reader about the current state of
knowledge. In conducting this research, we have used various types of
information, from both academic \footnote{Academic journals, academic
  theses} and non-academic \footnote{Websites, official reports issued
  by research or governmental organizations, magazines, etc.}
literature. The selection of papers was done by taking into account
the topic of investigation, while the afterwards information has been
grouped by the types of risk. In our search we have used a large variety
of keywords such as: crypto, Blockchain, financial risk, technological
risk, attack, financial behavior, Blockchain literacy, etc. With this
literature survey, we will answer to the following research question:
‘Can financial risks be triggered by technological vulnerabilities of
Blockchain technology?’.


With the aim to enlighten our research problem, the objective of this
survey is to provide a two-dimension risk analysis (technological and
financial) completed by an assessment of triggering elements (the
likelihood). Furthermore, following the example of \cite{Benoit2017}
literature survey, we are going to complete this study with a short
data analysis. In line with the statements made in the literature
review, we show that bitcoin’s price instability (financial risk) can
be triggered by the attacks targeting the crypto-market (technological
vulnerability).


This paper is organized as follows. Section 2 presents the assessment
of technological and financial risks; Section 3 proposes a brief
empirical illustration. Section 4 discusses the results and
concludes.



\section{Blockchain risks assessment}

In this section we perform a theoretical risk assessment of the
crypto-market. The goals of this assessment are:  \\

 \begin{tabular}{@{$\bullet$ }ll}
&To understand the vulnerabilities of Blockchain and their possible consequences and impact; \\
&To offer a broad view to Blockchain stakeholders on possible financial and technological risks.\\
\end{tabular} \\ \\
According to \cite{Leemoon2017}, crypto-market’s challenges can be divided into four main areas:
\begin{enumerate}
  \item Technological issues
  \item Financial issues
  \item Policy and legal issues
  \item Political issues
\end{enumerate}

While all four types of risks are indisputably affecting the
crypto-market development and slowing its acceptance, we consider that
the first two could represent a starting point and a reliable support
in designing a better legal framework. At the same time, we believe
that all these together could eventually alleviate some of the
political issues. That being said, in this study, we tackle the first
two categories, leaving the last two for future research. We make a
parallel analysis between the technological and financial risks. At
this point, we, for instance, intend to help users and investors find
the answer to possible questions, such as: “Is this investment /
technology safe? What are the risks and vulnerabilities I may
encounter?". To the best of our knowledge, this is the first study
trying to assess both financial and technological risks in parallel.


The complexity of this technology, inherited by nature, represents a
challenge for users, investors and any other participants from this
market \citep{Salmela2019}. Highly secure at first sight, Blockchain
is not exempt of risks, but is rather an imperfect innovation leaving
generous room for many improvements \citep{Iwamura2019}. According to
\cite{Swan2017} Blockchain technology is the only one which has the
potential to change or, better said, to revolutionize the way
businesses and financial markets work.


According to the latest surveys performed with regards to the factors
slowing down the Blockchain’s adoption, the main barriers are: the
scalability issues, insufficient regulation, unproven or the debatable
value of technology, security threats, lack of in-house skills and
uncertain rate of return \citep{UnderscoreVC2018, Pawczuk2019}. As we
can observe, most of the mentioned obstacles are either technological
or finance related. These findings encourage us to perform a risk
assessment and support the necessity of prioritizing the first two
categories of risks, namely the financial and technological ones.



\subsection{Technological risks}
We here systemize the crypto-market threats in accordance with their
nature, namely, consensus level attack, network level attack,
cryptographic key attacks and smart contract attacks. There are many
types of attacks which are not discussed in this study. However, we
tried to cover the most important ones, by taking into account the
likelihood, the exposure of the crypto-market to such incidents and
the impact they might have.


\textbf{Consensus algorithms} for Blockchain technology represent a
code-based protocol, aiming to facilitate reaching agreement processes
within a network. These algorithms came as a solution to the
“Byzantine General Problem”, which concerns the failure of reaching
consensus due to faulty actors \citep{Zhang2019}. The most popular and
widespread consensus algorithms in the Blockchain technology, are the
Proof of Work (PoW), Proof of Stake (PoS) and the Practical Byzantine
Fault Tolerance (PBFT) protocols (see Table \ref{consensus}).



\begin{table}[h!]
\centering
\caption{\textbf{Comparison of most notable consensus mechanisms used in the Blockchain applications}}
\label{consensus}
\begin{tabular}{||c c c c||}
 \hline
\textbf{Proprieties} & \textbf{PoW} & \textbf{PoS} & \textbf{PBFT}  \\ [0.5ex]
 \hline
 \textbf{Blockchain type} & \textit{Permissionless} & \textit{Permissionless} & \textit{Permissioned} \\
\hline
  \textbf{Fault Tolerance} & \textit{$<$50\%(of computing power)} & \textit{$<$50\%(of stake)} & \textit{$<$33\%(of faulty nodes)} \\ [1ex]
 \hline
\end{tabular}

\end{table}

The most noteworthy \textbf{attacks at the consensus level}, are:

\underline{Nothing at stake attack:} on the PoS protocol, where low
stake owners try to decrease the value of cryptocurrency. Indeed, the
control inside the system is given based on the user’s wealth,
potentially combined with other factors (coin age-based selection or
random factors). Any PoS Blockchain can be exposed to this type of
attack, especially in their beginnings, when there are no real
imbalances among the users’ wealth and low stake owners will not lose
much \citep{Morganti2019}.


\underline{The majority attack ($>$50\% attack):} means that the
consensus protocol is compromised, functioning as a monopolistic
system. Taking into account its possible implications, the majority
attack is considered also a security issue. Moreover, considering the
target type, it can be split in two variants: “the $>$50\% (or 51\%)
computational power attack” \footnote{an attack on the PoW protocol,
  implying the possession of more than 50\% of the total mining power,
  with the purpose to manipulate and corrupt the network} and “The
51\% stake attack” \footnote{An attack targeting the PoS protocol; it
  implies the possession of more than 50\% of the total circulating
  supply of coins (within the same network) with the purpose to gain
  monopoly power and mislead the system for profit purposes. It is
  conceptually similar to computational power attack.}
\citep{Tuwiner2021, Blockchain.com2020}.


Bitcoin has never experienced a successful majority attack. However,
we cannot say the same about altcoins: Feathercoin (June 2013),
Bitcoin Gold (May 2018), Vertcoin (December 2018), Ethereum Classic
(January 2019) and Bitcoin Cash (May 2019) \citep{Beigel2019}. The
difficulty to execute an attack is very much influenced by the size of
the Blockchain network. Table \ref{attacks} shows how expensive is to
perform a majority attack, depending on the cryptocurrency. These
costs are computed taking into account the expenses incurred in the
mining process, namely the network hash rate \& the Nicehash cost in
BTC /per hour (rented PC power). These values can change every minute,
as the cryptos' prices have a high influence \citep{Crypto51.app2020}.


\begin{table}[h!]
\centering
\caption{\textbf{PoW 51\% attack cost for the top 7 cryptocurrencies.}}
\begin{tabular}{||c c c ||}
 \hline
\textbf{System} & \textbf{Hash rate \footnote{The Hashing power is expressed in different units: TH = TeraHash; MH = MegaHash; KH = KiloHash; GH = GigaHash \citep{CoinGuides.org2020}.}} & \textbf{1 h attack estimated Cost}  \\ [0.5ex]
 \hline
 \textbf{Bitcoin} & \textit{114,915 PH/s} & \textit{\$716,072} \\
\hline
  \textbf{Ethereum} & \textit{253 TH/s} & \textit{\$418,438} \\
\hline
  \textbf{Litecoin} & \textit{227 TH/s} & \textit{\$29,287)} \\
\hline
  \textbf{B. Cash} & \textit{1,374 PH/s} & \textit{\$8,560)} \\
\hline
  \textbf{Zcash} & \textit{8 GH/s} & \textit{\$8,710} \\
\hline
  \textbf{B. SV} & \textit{1,109 PH/s)} & \textit{\$6,912} \\
\hline
  \textbf{Dash} & \textit{7 PH/s)} & \textit{\$3,246}  \\ [1ex]
\hline
\end{tabular}
 \\ \footnotesize{\textit{Values computed as per 10th February 2021}} \\
Source: derived from \cite{Crypto51.app2020}
\label{attacks}
\end{table}


\textbf{Network level attacks} are widely considered difficult and
expensive to perform \citep{Koshik2019}, however, they should never be
regarded as impossible.


\underline{DDoS (Distributed Denial of Service:} refers to an attack
on the host, aiming to disrupt the normal operation process. If for
example, the (host) Blockchain system is under attack, it can become
unresponsive, unavailable. The system is compromised by being feed
with misleading information or large amounts of data
\citep{Zhang2019}. DDoS attacks can have a notable impact within the
crypto-market, as they can target Blockchains \footnote{The difficulty
  to execute an attack is very much influenced by the size of
  Blockchain network. Private Blockchains are considered more exposed
  compared to the public ones, as they usually grow around just 100
  nodes. The adversary needs to control only 33\% of the network to
  perform an attack, which is easier to achieve in small Blockchains
  \citep{Saad2019}.}, exchange and trading platforms and even mining
pools \citep{Abhishta2019, Litecoinpool.org2020}. These attacks are
highly associated with the increase in value and popularity of the
cryptocurrencies \citep{Crothers2021}.


Some other notable examples of network level attacks, worth to mention
if we take into account the exposure and powerful impact they could
have, are the Sybil attack \footnote{a user creates multiple
  identities and uses them to gain dominance and manipulate the
  Blockchain system} and the Eclipse attack \footnote{similar to a
  Sybil attack, Eclipse misleads its victims such as they will see and
  believe a different truth than the rest of the network}. From our
knowledge, there is no Sybil or Eclipse attack successfully performed
on the Blockchain technology, in practice, but researchers have made
theoretical demonstrations for the Eclipse attacks on both PoW (Ether
and Bitcoin) \citep{Heilman2015,Packtpub2019, Marcus2018, Wust2016}
and PoS networks \citep{Zhang2019}. Usually, the network level attacks
are planned so they can precede other assaults
\citep{Morganti2019}. \\


\textbf{Cryptographic key attacks.} In Blockchain technology,
cryptographic keys give access to funds (through crypto wallets) and
play a critical role in transactional processes. In other words,
anyone handling the cryptographic keys can access the wallet account
and freely manage the associated funds. These keys are stored in
crypto wallets. According to the version of crypto wallet used
(software, hardware, cloud, brain\footnote{It’s a type of wallet which
  gives the user the option to generate a key using a password (a
  word, number, combination of bot, etc.). This type of wallet and
  keys are considered weak in terms of security.} or paper), the keys
are more or less safe (hardware \& paper - most secure, software,
brain \& cloud – less secure). Having such a variety of key storage
options gives attackers ideas to approach the wallets in different
ways.


\underline{Wallet attack:} The main causes behind wallet attacks are
system hacking, software vulnerabilities, malwares or incorrect usage
from the users’ side. The objective is to obtain (steal) the private
key, with which the attacker can mislead the system, perform
un-authorized transactions and steal coins (send them into the thief’s
wallet using the victim’s private key). Compared to any other types of
crypto attacks, the ones targeting the wallets are among the most
common and harmful incidents \footnote{In 2018 Coincheck’s wallets
  were hacked and lost \$530 million worth of NEM. This incident
  surpasses even the losses of Mt Gox case, being classified as the
  largest theft in the crypto history \citep{Shane2018}.}. This
statement is also supported by the Blockchain Graveyard organization,
as according to their thorough analysis on the incidents associated to
Blockchain, more than half relate to wallet attacks
\citep{Magoo.github.io2020}.


Some other notable examples of attacks at this level, are: the
\underline{Random number generator attack} \footnote{targets the weak
  security of the cryptographic keys, due to insufficient randomness
  used in their generation process, making them easy to predict
  (Independent Security Evaluators, 2019); in spite of the common
  knowledge that the cryptographic keys are difficult to break,
  apparently, a combination of weak hashing algorithms and skilled
  hackers have led to such kind of incidents.} and \underline{Quantum
  attacks} \footnote{performed with the quantum computers (QC); In the
  context of Blockchain, they can break the cryptographic keys,
  corrupt the hashing functions and forge digital signatures. These
  attacks can have serious implications for the Blockchain network,
  implying theft of the users’ funds, crypto wallets corruption,
  dominance over the network and even possible recreation of the
  entire Blockchain. It is maybe a matter of time, until we will have
  a QC powerful enough able to break the Blockchain technology
  \citep{Fernandez-Carames2020, Stewart2018}.} . \\


\textbf{Smart contract attacks} mainly refer to the manipulation of
external data entered in the Blockchain (through oracle technology),
misleading the execution of the smart contract. The trigger represents
information related to external events, which affects the contract’s
conditions. The information is manually introduced, reason why, the
execution of the system can be easily misled. Blockchain is an
open-source technology, giving access to its full code. This is an
opportunity for intruders, who may take advantage of this feature and
exploit it with malevolent intentions. Concurrently, if the
programming language used in the smart contract has weaknesses, this
might also create the perfect opportunity for any hacker to initiate a
successful attack \citep{Hasanova2019, Atzei2017}.


\underline{Re-entrancy attack}, as a variant, refers to a malfunction
in the smart contract protocol. During the attack, the hacker is
sending multiple requests to the system, as for example, invoking the
call function continuously until the gas supply ends. Overwhelmed by
the avalanche of orders, the system will perform inaccurately
\citep{Hasanova2019}.



A summary of all technological risks discussed above will be presented in Table \ref{tec}.


\newcommand{\spheading}[2][10em]{% \spheading[<width>]{<stuff>}
  \rotatebox{90}{\parbox{#1}{\raggedright #2}}}


%\begin{landscape}
\begin{sidewaystable}[htbp]
\centering
\small
\caption{\label{tec}\textbf{Summary of technological risks}}
%\resizebox{1.65\textwidth}{!}{
\begin{tabulary}{\linewidth}{p{25pt}|L|p{200pt}p{175pt}}
%\begin{tabularx}{\textwidth}{|c|lXX}
\hline
 &
  \centering{\textbf{Risk}} &
   \centering{\textbf{Consequences}} &
   {\textbf{Exposure}}  \\
\hline
\rowcolor[HTML]{FFFFFF}
{\multirow{7}{*}{\rotatebox[origin=r]{90}{\cellcolor[HTML]{FFFFFF}\textbf{Consensus level attack}}}} &
 {{\cellcolor[HTML]{EFEFEF}\textbf{Nothing   at stake attack}}} &
 {\cellcolor[HTML]{EFEFEF}\textit{· Manipulates the system by entering
     invalid data  \mbox{· Monopolized consensus process}}} &
  {{\cellcolor[HTML]{EFEFEF}\textit{Blockchains using PoS (over 350 cryptocurrencies\footnote{The total number of cryptocurrencies is over 2200 \citep{CryptoSlate.com2020}}) source:\citep{CryptoSlate.com2020}}}}
 \\
&
  {\cellcolor[HTML]{FFFFFF}} &
  {\cellcolor[HTML]{FFFFFF}\textit{·  Manipulates the system}} &
  {\cellcolor[HTML]{FFFFFF}\textit{Blockchains using PoW consensus (over 500 cryptocurrencies);}} \\
 &
  {\cellcolor[HTML]{FFFFFF}} &
  {\cellcolor[HTML]{FFFFFF}\textit{·  Monopolized consensus process}} &
  {\cellcolor[HTML]{FFFFFF}\textit{Blockchains using PoS (over 350 cryptocurrencies)source: \citep{CryptoSlate.com2020}}} \\
 &
  {\cellcolor[HTML]{FFFFFF}} &
  {\cellcolor[HTML]{FFFFFF}\textit{· Enters invalid data in the system}} &
  {\cellcolor[HTML]{FFFFFF}} \\
 &
  {\cellcolor[HTML]{FFFFFF}} &
  {\cellcolor[HTML]{FFFFFF}\textit{· Forks the Blockchain}} &
  {\multirow{-2}{*}{\cellcolor[HTML]{FFFFFF}\textit{Mining pools\footnote{The total number of mining pools is not known, as there are many which keep their identity secret. Therefore, we cannot accurately assess the market exposure with this respect.}}}} \\
 &
  {\multirow{-5}{*}{\cellcolor[HTML]{FFFFFF}\textbf{Majority attack}}} &
  {\cellcolor[HTML]{FFFFFF}\textit{·          Performs other attacks (Eclipse, double spending, DoS)}} &
  {\cellcolor[HTML]{FFFFFF}\textit{}} \\ \hline
&
  {\cellcolor[HTML]{EFEFEF}} &
  {\cellcolor[HTML]{EFEFEF}\textit{·            Manipulates the system by entering invalid or large flow of data}} &
  {\cellcolor[HTML]{EFEFEF}\textit{All Blockchains (small ones most exposed)}} \\
\rowcolor[HTML]{EFEFEF}
{\cellcolor[HTML]{FFFFFF}} &
  {\cellcolor[HTML]{EFEFEF}} &
  {\cellcolor[HTML]{EFEFEF}\textit{·          Disrupts the normal operation process}} &
  {\cellcolor[HTML]{EFEFEF}\textit{Mining pools}} \\
 &
  {\multirow{-3}{*}{\cellcolor[HTML]{EFEFEF}\textbf{DDoS   attack}}} &
  {\cellcolor[HTML]{EFEFEF}\textit{·          Knocks out part of or the whole network}} &
  {\cellcolor[HTML]{EFEFEF}\textit{Exchange platforms}} \\
\rowcolor[HTML]{FFFFFF}
{\cellcolor[HTML]{FFFFFF}} &
  {\cellcolor[HTML]{FFFFFF}} &
  {\cellcolor[HTML]{FFFFFF}\textit{·            Manipulates the system}} &
  {\cellcolor[HTML]{FFFFFF}} \\
\rowcolor[HTML]{FFFFFF}
{\cellcolor[HTML]{FFFFFF}} &
  {\cellcolor[HTML]{FFFFFF}} &
  {\cellcolor[HTML]{FFFFFF}\textit{·          Monopolized consensus process}} &
  {\cellcolor[HTML]{FFFFFF}} \\
\rowcolor[HTML]{FFFFFF}
{\cellcolor[HTML]{FFFFFF}} &
  {\multirow{-3}{*}{\cellcolor[HTML]{FFFFFF}\textbf{Sybil   attack}}} &
  {\cellcolor[HTML]{FFFFFF}\textit{·          Enters invalid data in the system}} &
  {\multirow{-3}{*}{\cellcolor[HTML]{FFFFFF}\textit{Permissionless   Blockchains}}} \\
\rowcolor[HTML]{EFEFEF}
{\cellcolor[HTML]{FFFFFF}} &
  {\cellcolor[HTML]{EFEFEF}} &
  {\cellcolor[HTML]{EFEFEF}\textit{·            Manipulates the system}} &
  {\cellcolor[HTML]{EFEFEF}} \\
\rowcolor[HTML]{EFEFEF}
{\cellcolor[HTML]{FFFFFF}} &
  {\cellcolor[HTML]{EFEFEF}} &
  {\cellcolor[HTML]{EFEFEF}\textit{·          Monopolized consensus process}} &
  {\cellcolor[HTML]{EFEFEF}} \\
\rowcolor[HTML]{EFEFEF}
{\multirow{-9}{*}{\rotatebox[origin=c]{90}{\cellcolor[HTML]{FFFFFF}\textbf{Network level attack}}}} &
  {\multirow{-3}{*}{\cellcolor[HTML]{EFEFEF}\textbf{Eclipse attack}}} &
  {\cellcolor[HTML]{EFEFEF}\textit{·          Enters invalid data in the system}} &
  {\multirow{-3}{*}{\cellcolor[HTML]{EFEFEF}\textit{Permissionless   Blockchains}}} \\ \hline
\rowcolor[HTML]{FFFFFF}
{\cellcolor[HTML]{FFFFFF}} &
  {\cellcolor[HTML]{FFFFFF}} &
  {\cellcolor[HTML]{FFFFFF}\textit{·            Steals the cryptographic keys}} &
  {\cellcolor[HTML]{FFFFFF}} \\
\rowcolor[HTML]{FFFFFF}
{\cellcolor[HTML]{FFFFFF}} &
  {\cellcolor[HTML]{FFFFFF}} &
  {\cellcolor[HTML]{FFFFFF}\textit{·          Takes the control of the afferent funds}} &
  {\cellcolor[HTML]{FFFFFF}} \\
\rowcolor[HTML]{FFFFFF}
{\multirow{7}{*}{\cellcolor[HTML]{FFFFFF}\spheading[50pt]{\textbf{Cryptographic key threats}}}} &
  {\multirow{-3}{*}{\cellcolor[HTML]{FFFFFF}\textbf{Wallet attack}}} &
  {\cellcolor[HTML]{FFFFFF}\textit{·          Deters the security and trust of the users}} &
  {\multirow{-3}{*}{\cellcolor[HTML]{FFFFFF}\textit{All Blockchains}}} \\
 &
  {\cellcolor[HTML]{EFEFEF}\textbf{Random   number generator attack}} &
  {\cellcolor[HTML]{EFEFEF}\textit{·  Corrupts the cryptographic keys \& crypto wallets}} &
  {\cellcolor[HTML]{EFEFEF}\textit{All Blockchains}} \\
 &
  {\cellcolor[HTML]{FFFFFF}} &
  {\cellcolor[HTML]{FFFFFF}\textit{·            Corrupts the cryptographic keys \& crypto   wallets}} &
  {\cellcolor[HTML]{FFFFFF}} \\
 &
  {\cellcolor[HTML]{FFFFFF}} &
  {\cellcolor[HTML]{FFFFFF}\textit{·          Forges hashing functions \& digital signatures}} &
  {\cellcolor[HTML]{FFFFFF}} \\
&
  {\multirow{-3}{*}{\cellcolor[HTML]{FFFFFF}\textbf{Quantum attacks}}} &
  {\cellcolor[HTML]{FFFFFF}\textit{·          Rewrites Blockchain and manipulation of the network}} &
  {\multirow{-3}{*}{\cellcolor[HTML]{FFFFFF}\textit{All Blockchains}}} \\ \hline
\rowcolor[HTML]{EFEFEF}
{\multirow{2}{*}{\cellcolor[HTML]{FFFFFF}\spheading[50pt]{\textbf{Smart
        contract threats}}}}
 &
  {\cellcolor[HTML]{EFEFEF}\textbf{Reentrancy   attack}} &
  {\cellcolor[HTML]{EFEFEF}\textit{Manipulates the network   \& spends unlimited}} &
  {\cellcolor[HTML]{EFEFEF}\textit{Blockchains   supporting smart contracts (over 50 cryptocurrencies) Source:\citep{CryptoSlate.com2020}}} \\
 &
  {\cellcolor[HTML]{FFFFFF}\textbf{Smart contract attack}} &
  {\cellcolor[HTML]{FFFFFF}\textit{Misleads the technology’s application}} &
  {\cellcolor[HTML]{FFFFFF}\textit{Blockchains supporting smart contracts (over 50 cryptocurrencies) Source:\citep{CryptoSlate.com2020}}} \\ \hline
\end{tabulary}
%}
\end{sidewaystable}

%\end{landscape}





\subsection{Financial risks}

 In this section, we give example of several financial risks that can
 be triggered by technological risks. After detailing how this
 phenomenon happens and in what kind of circumstances, we propose a
 conceptual metric, with the purpose to emphasize the likelihood that
 these technological risks may transform into financial ones.


\underline{Determining the likelihood:} The likelihood that the
technological risks may transform into financial risks, can be
established by taking into account the severity \footnote{Financial
  loses and investment cost incurred.} effect and probability of
occurrence of triggering elements. Here, we will also introduce the
concepts financial behavior, responsible investment and Blockchain
literacy, as possible tools for assessing risk. Measurement plays an
important role in management. Up to this point, we have different
tools to measure financial risks, however, things are not as simple
when talking about the triggering elements. According to
\cite{Kaplan1992},if we can’t measure something, then we can’t
properly manage it. Therefore, in this part of the assessment, we
propose ways to measure the probability that technological
vulnerabilities may trigger financial risk. \\


\textbf{Total market risk.} This is the financial risk arising from
high movement in market prices. The most used measure for appraising
the total market risk of an asset is the volatility of its market
returns. Following the traditional financial theory, the total market
risk can be decomposed into the systematic risk and the specific
one. If the crypto-market is vulnerable to a risk threating the whole
market, this could be a systematic risk. On the other hand, if we
consider risks targeting a specific crypto-asset or type of
Blockchain, then this could be an example of specific risk
\footnote{Specific risk concerns isolated cases (one crypto-asset or a
  specific group, usually not dominating the market) and has fewer
  casualties than a systematic risk, which affects a large part of the
  market or the whole.}.



 From the above list, by taking into consideration the risks’ exposure
 and their consequential power, we can easily identify several
 attacks, which may trigger financial risks. For instance, majority
 attacks (almost half of the total crypto-market is exposed to this
 risk, plus the mining pools), Sybil and Eclipse attacks (targeting
 Permissionless Blockchains- the most common and largest representatives
 of this market-), DDoS attack, wallet attack, random number generator
 attack and quantum attacks (targeting all types of Blockchains) can
 be considered potential triggers for systematic risk \footnote{risks
   inherent to the entire market or market segment, reflecting not
   just the impact of economic, geo-political and financial factors
   but also the technological vulnerabilities.}. At the same time, if
 affecting just one type of Blockchain, one cryptocurrency or few
 casualties such as a mining pool/exchange platform, the same
 technological can trigger a specific risk.


 It is well known that the crypto-assets’ price is influenced by
 regulatory and cybersecurity related events
 \citep{Corbet2019}. Subsequently, such events influence the
 investors’ behavior, which is eventually impacting crypto-market’s
 volatility. It was also proved that cryptocurrencies suffer from
 contagion effects (herding behavior)\citep{DaGamaSilva2019}. Bitcoin,
 Ether or any other strong and well-known currency have proved their
 influence over the evolution of the whole cryptocurrency market. In
 2017, when Bitcoin prices skyrocketed and crashed, the rest of the
 cryptocurrencies followed a similar trend \citep{Antonakakis2019,
   Pereira2019}. The strong power of influence and the herding
 behavior present in the crypto-market, represent a trigger for
 systematic risk. Here, we have the perfect example of how an
 independent event, initially affecting one currency (specific risk),
 can eventually transform in a systematic risk \footnote{this was
   possible through investor’s behavior, which tend to associate
   Bitcoin‘s image with the one of the whole market.}, impacting the
 whole market \citep{Jain2019}. It is well known that, systematic risk
 can be triggered by various factors such as socio-political, economic
 and any other market-related events. In the crypto-market, we can see
 that on top of the already existing factors, we have also the
 technological vulnerabilities as a possible trigger. Under the
 hypothesis of traditional financial theory, specific risk is
 diversifiable and is not priced by the market. On the opposite,
 investors require a risk premium, and thus, higher returns for
 compensating the systematic risk they incur.


\underline{Likelihood:} The main triggers for market risks are the
cyber-attacks and technological risks. According to
Blockchain-Graveyard database of crypto attacks, the most frequent and
damaging are the ones on cryptographic keys (about half of the total
incidents), followed by application vulnerabilities (security
breaches) and protocol issues \citep{Magoo.github.io2020}. As a
vicious circle, good financial conditions in the crypto-market can
motivate intruders to perform more attacks
\citep{Crothers2021}. Eventually, depending on the amplitude of
damaged caused, technological risks might transpose into different
financial risks. Since attacks are pretty common in the crypto-market
and usually imply important financial losses, we state that the
likelihood as high. \\


 \textbf{Information risk} risk refers to the imbalance of information
 spread among the market players. Conceptually speaking, thanks to its
 features, Blockchain technology represents itself a useful tool in
 reducing information asymmetry, assuring transparency and
 trust. However, along the evolution of the crypto-market, these
 innovations became more complex, challenging investors and users to
 acknowledge the potential. The novelty and technological nature of
 the crypto-market may get stakeholders into trouble, as some do not
 understand it. At the same time, the lack of knowledge and specific
 skills, sometimes completed by the insufficient information supplied
 to the public, increases the uncertainty and restrain towards the
 whole market.


Compared to any other Blockchain application, initial Coin Offerings
(ICO) impose most of the problems regarding the transparency and
information asymmetry. The complexity of ICOs’ white paper \footnote{a
  document describing the technology used in the Blockchain project
  (ICO). It has the purpose to convince the public that the new
  crypto-asset offers a good investment opportunity.}, investors’ lack
of training and the insufficient regulation, led to manipulation and
financial losses for investors. According to the existing literature,
most investors in this market, lack the required capabilities to
interpret the market’s signals. The discrepancy between traditional
market and crypto-market, pushes investors and users towards
questionable sources of information such as social media. Here, the
selection is based rather on the ‘easy-to-interpret' criteria than
quality and credibility. At the same time, the general opinion
surrounding the crypto-market seems to influence the players
(investors and users), which might take decisions rather based on the
social trends (led by a herd mentality \footnote{a “If I am losing, at
  least I am not losing alone’ mentality – investors might believe
  that following trends or the majority provides some security and
  makes losses easier to tolerate \citep{Gazali2018}}.) than
rationally. This could explain the inefficiency of the crypto-market,
despite the quantity of information available \citep{RuiChen2020,
  Gazali2018}.


\underline{Likelihood:} Among the most important factors responsible
for information risk in the crypto-market, we have the lack of
available information (e.g. white / yellow papers, inconsistent data)
and insufficient knowledge or understanding for investors and
users. Due to the poor regulatory framework, intruders found an
opportunity to become rich overnight. They issue low quality
crypto-assets, about which there is little information available
(incomplete white papers or inconsistent data), and use them to trick
the other market players. This risk is behind most of the fraudulent
coins or low-quality ICO projects. Reputation might attract more
enthusiasts in this market; therefore, we believe that the investors
interested in cryptos are pretty various. Here, we introduce the
Blockchain literacy (ability to understand the Blockchain related
knowledge and make informed and effective decisions
\citep{VanRooij2011})and financial behavior (how individuals gather
and interpret information, eventually reflecting in decisional
processes \citep{DeBondt2008}), concepts, as important factors in the
way the market evolves. Market signals can be complex, including both
information and noise \citep{Rizzi2008}. Less mysterious than at the
beginning, however, still significantly complicated, the Blockchain
world might pose some problems in understanding. Blockchain illiteracy
leads to irrational behavior, which eventually reflects in inefficient
markets. Taking into account the large number of crypto scams and the
important financial losses incurred (especially during the Bitcoin
bubble 2017-2018 \citep{Zetzsche2019, Liebau2019}), we state that the
likelihood for this risk is high. \\


\textbf{Liquidity risk.} A market is said to be liquid if an agent can
make rapidly some significant trades without creating an important
change in the price (small market impact). In other words, in a liquid
market, transactions will likely not create a change in the price, but
new information will be smoothly incorporated. On the other hand, an
illiquid market (most of the time linked to inefficient market), will
reflect a large volatility in prices (hence a higher probability of an
unfair price), a lower number of investors and lower chances to
transact/trade.


 Liquidity risk, can be split into three categories: assets liquidity
 (refers to the interaction between sellers and buyers on the platform
 and the asset availability on exchanges), exchange liquidity (refers
 to the interaction between makers and takers concerning the assets’
 and the orders’ supply) and market liquidity (encompasses the first
 two) \citep{Crowell2020}. The most debated factors explaining
 liquidity in the crypto-market are the price, trading volume,
 capitalization, fees, hash value (for PoW cryptocurrencies) and the
 size of the network. Contrary to traditional assets, in the
 crypto-market, high returns are negatively correlated with liquidity,
 while a rise in trading volume, market capitalization and volatility
 are associated with lower liquidity uncertainty
 \citep{Koutmos2018}. At the same time, liquidity risk is highly
 correlated with the events concerning cyber-attacks or regulatory
 issues, as a response to human behavior and investors’ attitude
 towards this market \citep{Corbet2019}.

A liquid market will be stable, showing less volatility and a larger
range of orders to pick from. A stable market in a liquid environment
is resistant to possible manipulation, such as whales or group orders,
placed with the intention to exploit the price benefits. It is
important to mention the fact that liquidity is different from one
cryptocurrency to another (the most popular ones are more liquid) as
well as from one exchange platform to another. In spite of the many
benefits associated with liquidity, illiquid environments can also
present some advantages, especially for the traders for this market,
which can benefit from new arbitrage opportunities and purchases at
discounts \citep{Crowell2020}.


\underline{Likelihood:} Analyzed from the cryptocurrencies’
(crypto-assets that claim to be ‘money’) perspective, this risk would
translate into an impossibility to be transformed in cash. That being
said, one of the principal roles of money (being a medium of exchange)
has just failed \citep{Greene2018}. There are many triggers behind
crypto-assets illiquidity, among which: token supply algorithm,
investors’ behavior, available supply, asset usage, fees, exchange
platforms failure, etc. As liquidity risk is already well-known in the
financial markets (is one of the indicators for market efficiency), we
already know tools to measure it (trading volumes, book depth and the
bid-ask spread, different liquidity ratios, etc.)
\citep{Jain2018}. Similar to traditional securities, crypto-market
suffers from illiquidity during extreme price movement period of times
\citep{Manahov2020}. A proof of market efficiency, represents the
difficulty to manipulate prices. In the crypto-market, specifically
concerning bitcoin, it has been observed a significant hoarding
behavior. The number of bitcoin whales increasing to the impressive
number of more than 2 thousand addresses \footnote{Owning between
  1,000 to 10,000 BTC.} \citep{Bitcoin.com2020}. Beside the fact that
hoarding implies a significant movement in prices (buy/sell large
amounts of crypto-assets), it has important supply implications as in
the end, there are little assets available to trade
\citep{Manahov2020}. Asset usage plays an important role within this
market, as the more people believe the assets has value, the more
desirability to trade it. The asset usage perception is increasing
along with the acceptance and development of crypto-market. Liquidity
is an important characteristic of the market, influencing the
investment costs and implicitly the desirability to trade. If we look
at this risk from the Bitcoin perspective, we could easily state that
liquidity risk is very high. At the same time, by capturing the big
picture of the crypto-market, where we have over 2000 crypto-assets
available, we state that the likelihood is medium. \\


\textbf{Supply risk} refers to the reserve available of
crypto-assets. Some of examples of important supply risk triggers are
the loss of cryptographic keys (without which there is no possibility
to access the afferent funds), cyber-attacks\footnote{e.g. the coins
  may stay blocked in the intruder’s account for a while, attempting
  to avoid the public eye. }, unclaimed rewards
\citep{Coinmetrics.com2019}, reputation and the programmed limit of
supplies. Not all the cryptocurrencies have a maximum supply
limit. For example, cryptocurrencies such as Bitcoin, Ripple, IOTA,
Litecoin and many others, have a pre-established limited supply, while
coins like Ethereum, Zcash, Monero and others have no such
limits. Following Rational Expectation Equilibrium models, the higher
the supply uncertainty, the less informative crypto-assets prices will
be. In this case, market prices are less efficient and supply risk
could thus even lead to an information risk
\citep{Collomb2016}. Compared to national currencies, cryptocurrencies
(especially bitcoin) were conceived as being less sensitive to the
market changes and inflation rate. However, with time we saw that
Satoshi’s ‘perfect’ innovation leaves room for further improvement.


Mainly associated with market inefficiency at users’ and exchange
platforms’ cost, the supply risk is affecting the mining and
transaction validation processes, as well. Miners are absolutely
necessary in a PoW Blockchain performing both transaction validation
and coin ‘minting’ functions. For successful work, they are rewarded
by the system with an amount of newly created crypto coins. The reward
offered by the system, represents a method to create new coins and to
increase the available supply of cryptocurrencies. At the same time,
rewards are programmed to decrease steadily, until the maximum supply
will be reached \citep{Eyal2018}. When this happens, the mining reward
will be based only on transactions fees \citep{CryptoLi.st2020}.


Keeping in mind the above arguments, we state that the difficulty to
create (mine) new cryptocurrency, the supply limits and the expenses
incurred during this process, they all have a great impact on the
supply imbalances and the final value of the assets.


\underline{Likelihood:} Since market liquidity is driven by the total
supply available for trade, we understand that it makes it an
important characteristic for market efficiency as well. Among the most
notable triggers for supply issues, we have: token supply algorithm,
hoarding behavior, loss of keys, wallet attacks,
etc. \citep{Coinmetrics.com2019}. If the supply limits are not a risk
for all the crypto-assets, it represents a threat at market level,
concerning the leader bitcoin. As initially programmed, bitcoin
maximum supply is 21 million coins. The already issued coins attain
the approximate number of 18 million, supposing that the limit will be
reached sometime around 2140 \citep{Ciaian2015}. As we already
discussed the negative sides of limited supply (illiquidity and market
inefficiency), we will mention now the bright side of this
risk. Similar to commodities such as precious metals and natural gas,
crypto-assets with limited supply attain high preference (subsequently
high value), being regarded as ‘scare’ assets. By just looking at the
price and market share of bitcoin, we can obviously observe that the
investor’s choices show a specific preference for this coin. In this
case, the financial behavior within this market is under the influence
of ‘scarcity gives value’ idea \citep{Verhallen1982}. However, this
idea of value can bring important investment costs, as investors
putting their money into such assets, will consider asking for
scarcity premiums on top of the existing ones for other risks
\citep{Haase2013}.  By assessing the supply risk at crypto-market
level, we state that the likelihood is medium.\\


\textbf{Environmental risk.} Known as an energy-gourmet, Blockchain
technology represents one of the key players in the fight towards the
green transition \citep{Charles2019}. This type of risk concerns
specifically the PoW Blockchains, which through their design, require
high computational power and a lot of electricity, for functioning
purposes. According to recent surveys, the bitcoin network is
responsible for using about 0.2\% of the global electricity and
emitting as much carbon dioxide emission as the country of Jordan
\citep{Irfan2019}. Another important aspect to mention is the
increasing number of ICOs, which require Ethereum Blockchain (PoW
based) for their smart contract application. According to the current
statistics, there are over three hundred thousand ether derived
crypto-assets (both active and non-active\footnote{tokens from former
  ICOs.} tokens) \citep{CryptoSlate.com2020}. We believe that the
technological constraints regarding the electricity consumption should
receive priority consideration; perhaps very soon, the success of ICO
projects and the performance of businesses (using Blockchain
technology), will be influenced by the environmental
considerations. In the light of current environmental context, there
were many attempts to reduce the costs and unnecessary pollution,
although no significant progress was made so far \citep{Lasla2020,
  Saleh2021, Bentov2016, Lepore2020}. The emergence of mining pools,
the use of renewable energy (74\% of the used electricity is
renewable) and lightning network, the emergence of platforms for
renting mining power (e.g. Nicehash), first step towards a greener
crypto world. Although, we know that there is a long road until we
reach the point of zero-emission power \citep{Irfan2019}. A solution
to stimulate a rapid transition to eco-friendly Blockchains, could be
the implementation of a tax regime relative to the amount of energy
consumed or to the units of carbon emitted per transaction. In this
way, the crypto industry could become more aware of its environmental
impact, contribute to the domestic economy and hopefully, make an
effort to find the best alternative for both the ecosystem and
business \citep{Mecca2019, Goodkind2020}. Simultaneously, with the
increasing sensitivity of investors to social responsibility of their
investment \citep{Brown-Liburd2015}, the assets showing negative
environmental externalities may be submitted to boycott from
investors. The environmental risk thus translates into a financial
risk.


\underline{Likelihood:} We know that during specific economic
conditions (pandemics, financial crisis, war, etc.) the stability of
financial markets can be highly affected. At the same time, as we
learn from the past events, such as the 2008 financial crisis or COVID
pandemics, the most performant and least risky investments were the
socially responsible ones \citep{Lins2017, Singh2020,
  Palma-Ruiz2020}. Well-informed market players have concerns
regarding the enterprise risk management, financial performance and
considerations for the surrounding environments \citep{Ballou2006}. As
a strategy to decrease the risk exposure and make safer ‘investment
bets’, investors pay careful attention at what kind of assets they put
money in and make more socially responsible investments.


Once with the creation of crypto-derivatives and tokenized securities,
we can consider that the first step towards a convergence between
crypto world and traditional markets was done. Crypto derivatives can
now be traded on both exchange platforms and OTC market
\citep{DeribitInsights2020}. Brokers can switch from securities to
crypto-assets, or trade both. Regarding investment preferences, it was
noticed that during turbulent periods and for safety considerations,
investors tend to choose financial markets in the favor of
crypto-market \citep{Matkovskyy2019}. Taking into account the
investors preference for ‘safety bets’ and concerns about
environmental and social implications, we believe that a more
ecologically oriented Blockchain could significantly change the
overall ‘safety’ perception. If this kind of risk doesn’t have direct
financial losses, it impacts the investment profitability, increasing
the costs\footnote{E.g. A company issuing ICO projects, can be
  directly affected by the investors’ social considerations, which
  will reflect in the amount of funds raised or the price/value of
  their crypto-assets (lower)} for financing. As time passes,
investors give more attention to the crypto-market, therefore we
consider that for the moment the likelihood is Medium. At the same
time, we would like to mention that there are many chances that the
likelihood becomes high, if from technological point of view nothing
changes. \\


A summary of all financial risks discussed above will be presented in Table \ref{fin}.




\begin{table}[h!]
\centering
\caption{\label{fin}\textbf{Summary of financial risks}}
\setlength{\tabcolsep}{1pt}
\begin{tabulary}{\linewidth}{|c|L|L|c|}
\hline
\rowcolor[HTML]{FFFFFF}
\textbf{Risk}                                                                              & \textbf{Trigger}                                                                       & \textbf{Influence /  Consequences}                                                                        & \textbf{Likelihood}                              \\
\hline
\rowcolor[HTML]{EFEFEF}
\multicolumn{1}{|c|}{\cellcolor[HTML]{EFEFEF}}                                             & \textit{Cyber-attacks}                                                                 & \textit{•          Large loses for investors.}                                                              & \cellcolor[HTML]{EFEFEF}                         \\
\rowcolor[HTML]{EFEFEF}
\multicolumn{1}{|c|}{\cellcolor[HTML]{EFEFEF}}                                             & \textit{Technological risks}                                                           & \textit{•          A sign that the market is not stable and mature}                                       & \cellcolor[HTML]{EFEFEF}                         \\
\rowcolor[HTML]{EFEFEF}
\multicolumn{1}{|c|}{\cellcolor[HTML]{EFEFEF}}                                             & \textit{Regulatory mismatches}                                                         & \textit{•          Crypto assets trade with a risk premium relative to the risk they may incur}           & \cellcolor[HTML]{EFEFEF}                         \\
\rowcolor[HTML]{EFEFEF}
\multicolumn{1}{|c|}{\cellcolor[HTML]{EFEFEF}}                                             & \textit{Human behavior}                                                                & \textit{}                                                                                                 & \cellcolor[HTML]{EFEFEF}                         \\
\rowcolor[HTML]{EFEFEF}
\multicolumn{1}{|c|}{\multirow{-5}{*}{\cellcolor[HTML]{EFEFEF}\textbf{Total market risk}}} & \textit{Reputation}                                                                    & \textit{}                                                                                                 & \multirow{-5}{*}{\cellcolor[HTML]{EFEFEF}High}   \\
\rowcolor[HTML]{FFFFFF}
\cellcolor[HTML]{FFFFFF}                                                                   & \textit{Lack of available information (e.g. white / yellow papers, inconsistent data)} & \textit{•          Financial loses for uninformed investors.}                                             & \cellcolor[HTML]{FFFFFF}                         \\
\rowcolor[HTML]{FFFFFF}
\cellcolor[HTML]{FFFFFF}                                                                   & \textit{Lack of knowledge/ understanding}                                              & \textit{•          Assets trade at prices far from their fundamental value}                               & \cellcolor[HTML]{FFFFFF}                         \\
\rowcolor[HTML]{FFFFFF}
\multirow{-3}{*}{\cellcolor[HTML]{FFFFFF}\textbf{Information risk}}                        & \textit{Reputation}                                                                    & \textit{}                                                                                                 & \multirow{-3}{*}{\cellcolor[HTML]{FFFFFF}High}   \\
\rowcolor[HTML]{EFEFEF}
\cellcolor[HTML]{EFEFEF}                                                                   & \textit{Regulatory mismatches}                                                         & \textit{•          Less investors}                                                                        & \cellcolor[HTML]{EFEFEF}                         \\
\rowcolor[HTML]{EFEFEF}
\multirow{-2}{*}{\cellcolor[HTML]{EFEFEF}\textbf{Liquidity risk}}                          & \textit{Reputation}                                                                    & \textit{•          Less efficient market}                                                                 & \multirow{-2}{*}{\cellcolor[HTML]{EFEFEF}Medium} \\
\rowcolor[HTML]{FFFFFF}
\cellcolor[HTML]{FFFFFF}                                                                   & \textit{Technological issue (supply limits)}                                           & \textit{•          Deflation, which can be a problem if crypto-assets will work as a method of payment}   & \cellcolor[HTML]{FFFFFF}                         \\
\rowcolor[HTML]{FFFFFF}
\cellcolor[HTML]{FFFFFF}                                                                   & \textit{Cyber- attacks}                                                                & \textit{•          Less efficient market}                                                                 & \cellcolor[HTML]{FFFFFF}                         \\
\rowcolor[HTML]{FFFFFF}
\multirow{-3}{*}{\cellcolor[HTML]{FFFFFF}\textbf{Supply risk}}                             & \textit{Loss of cryptographic keys}                                                    & \textit{}                                                                                                 & \multirow{-3}{*}{\cellcolor[HTML]{FFFFFF}Medium} \\
\rowcolor[HTML]{EFEFEF}
\cellcolor[HTML]{EFEFEF}                                                                   & \textit{Technological issue (PoW)}                                                     & \textit{•          Damage for the environment}                                                            & \cellcolor[HTML]{EFEFEF}                         \\
\rowcolor[HTML]{EFEFEF}
\cellcolor[HTML]{EFEFEF}                                                                   & \textit{Reputation}                                                                    & \textit{•          Crypto assets trade with a risk premium relative to their environmental externalities} & \cellcolor[HTML]{EFEFEF}                         \\
\rowcolor[HTML]{EFEFEF}
\multirow{-3}{*}{\cellcolor[HTML]{EFEFEF}\textbf{Environmental risk}}                      & \textit{Lack of regulation}                                                            & \textit{}                                                                                                 & \multirow{-3}{*}{\cellcolor[HTML]{EFEFEF}Medium}\\
\hline
\end{tabulary}
\end{table}



\section{Data analysis}

In line with the literature review done in the previous sections, here
we are going to provide an example of how financial risk is linked to
technological vulnerabilities. More specifically, we are going to
assess if bitcoin’s volatility is affected by the events targeting the
crypto-market. This data analysis is an illustration that complements
the literature survey performed in this paper, without any intention
to transform it in an empirical study. In accordance with the
literature and with the aim to answer to our research question, we
establish the following hypotheses:\\


\noindent \textbf{H1:Bitcoin’s volatility is positively linked to the number of events targeting the crypto-market.} \\
\noindent \textbf{H2: Bitcoin’s volatility is positively linked to the amounts lost due to these events targeting the crypto-market.}
\\
We retrieved bitcoin prices from Thomson Reuters Eikon database and
the list of events targeting bitcoin has been taken from
\cite{Biais2020}. Many other researchers have used the same Eikon
database, among which we mention: \cite{Corbet2020, Aliu2020,
  Akyildirim2020} etc. In total, our dataset comprises 53 events (see
table \ref{events}), and the bitcoin’s historical price starts from
August 2011 to September 2021.\\


In order to verify whether technological events have an influence on
the perceived risk of cryptocurrencies, we investigate the
relationship between bitcoin’s volatility and the attacks on
bitcoin. We check for the relationship between the volatility and the
number of events, as well as for the one between volatility and the
amount (in term of bitcoin) lost in each event. Volatility is widely
regarded as a signal for risk in finance. Hence, we are looking for a
relationship between technological events (attacks) and financial risk
(volatility).


In the following section we are going to compute volatility using
standard deviation method. Our choice is justified by the scope of
this analysis: to demonstrate that there is a relationship between
bitcoin’s volatility and our events. The rationale behind choosing
these events as a proof of technological vulnerability is the
following: most of these events are attacks that were possible thanks
to the characteristics of this market. By the characteristics of this
market, we mean:


\begin{itemize}
\item Cryptocurrencies represent a virtual currency; they exist and
operate just in the online environment. This makes them the target of
cyberattacks that exploit any possible vulnerability of this
technology.
\item Cryptocurrencies’ users need cryptographic keys in order to
access their funds or to place transactions. These keys easily become
the source of attacks, when they are not kept safely or the code is
easy to break.
\item The identity protection (anonymity) offered by the Blockchain
technology attracted many enthusiasts, however, this feature makes it
almost impossible to catch the hackers / thieves.
\item The insufficient regulation and incertitude around
cryptocurrency world made them the perfect tool for the black markets;
these ones being also the few places accepting cryptocurrencies as a
payment.
\item Blockchain literacy is slowly increasing. Along the time, the
lack of proper understanding about how this crypto world works, was
exploited in many forms in order to trick the users and steal their
coins. An example would be the many scams performed by crypto-exchange
platforms.
\item Blockchain is a decentralized technology and its transactions
are immutable. That makes it impossible to reverse fraudulent
transactions or recuperate the stolen funds. This characteristic
together with the anonymity feature may incite malevolent actors to
execute their plans.
\end{itemize}

We compute the monthly standard deviation of bitcoin’s returns as:

$$ \sigma = \sqrt{\frac{(R - \mu)^2}{n}}$$

Where, R are bitcoin returns, $\mu$ is the average return and n is the number of days of the window considered. Accordingly with our hypotheses, we perform the following linear regression:
 $$ \sigma = \alpha + \beta * EVENT_{number} + \epsilon   $$
 $$ \sigma = \alpha + \beta * EVENT_{amount} + \epsilon   $$

Where, $EVENT_{number}$ is is the variable representing the monthly
volume of events targeting bitcoin and $EVENT_{amount}$ is the monthly
amounts lost, in bitcoins, due to these events.\\


The first regression checks if there is a relationship between the
monthly volatility of bitcoin and the number of events per month. The
results obtained are significant, with a p value of 0.0216. We
therefore conclude that our sample data provided enough evidence to
reject the null hypothesis; these results show that there is a
relationship between the monthly volatility of bitcoin and the number
of events targeting it. Detailed results are shown in appendix
section, table \ref{reg1}.



In the second regression, we test if there is a relationship between
the volatility level of bitcoin and the amount (in terms of bitcoin)
lost as a result of these events. The amounts have been aggregated on
monthly basis. The result obtained (p value of 0.66) show that there
is no relationship between the volatility and the amounts
lost. Detailed results are shown in appendix section, table
\ref{reg2}.



Furthermore, we measure the relationship degree between the volatility
and the number of events, using Pearson test and Spearman’s rank
correlation. Pearson, also known as a parametric correlation test, is
one of the most common methods used in assessing the degree of
relationship between two linearly related variables
\citep{Pearson1932}. Spearman rho (a non-parametric test) measures the
degree of association between two variables \citep{Spearman1904}. Both
tests confirmed that bitcoin’s volatility is correlated with the
number of events. The results can be seen in the bellow table
\ref{correlation}.



\begin{table}[h!]
\centering
\begin{tabular}{||c c c c||}
 \hline
\textbf{Test} & \textbf{p-value} & \textbf{Correlation estimates} & \textbf{Interpretation}  \\ [0.5ex]
 \hline
 \textbf{Pearson} & 0.02156 & 0.4402268 & Moderate positive correlation \\
\hline
  \textbf{Spearman} & 0.03387 & 0.4095827 & Moderate positive correlation \\
\hline
  \end{tabular}
\caption{\label{correlation}\textbf{Correlation tests for \em{the volatility versus number of events}}\\ \footnotesize{\textit{The p-values resulted from the correlation tests are less than the significance level alpha = 0.05. We can conclude that the monthly volatility of bitcoin and the number of events targeting it, are moderately correlated with correlation coefficients of 0.44 and 0.40.}}}
\end{table}

It is interesting to observe that according to our analysis, the
volatility of bitcoin is not influenced by the financial losses
inquired by the users of bitcoin, but rather by the number of attacks,
scams or other negative events targeting this market. This result
proves that the participants from the crypto-market are more sensitive
to cybersecurity risks than financial ones. A way to justify this
would be the discrepancy between the users’ expectations versus
reality. Blockchain technology was created with the purpose to offer a
more secure and transparent alternative to the existing payment
tools. However, that doesn’t make it immune to cyberattacks and
scams. At the same time, we argue that Blockchain illiteracy may have
an important implication in how crypto-market players behave.



\section{Conclusion}

The crypto-market emerged in 2008, together with the first
cryptocurrency created, bitcoin. Since then, Blockchain technology
evolved, potentially disrupting many fields beyond finance. However,
it is still in infancy compared to its promised future, the
crypto-market has to overcome its many challenges. We believe that
understanding and analyzing the crypto-market vulnerabilities,
represents the first step in overcoming its challenges. \\
In this paper we perform a literature survey focusing the types of
risks present in the crypto-market. Our focus is on the technological
and financial risks of crypto-market and Blockchain technology. First,
we show that these risks can be related and that during specific
market conditions, they can become a trigger one for another. From our
knowledge, this is the first study showing that financial risks can be
triggered by the technological vulnerabilities of Blockchain. Second,
we offer a way to determine the likelihood of triggering financial
risks through technological vulnerabilities. Here, we also emphasize
the role played by financial behavior, social responsibility and
Blockchain literacy in the stability of crypto-market. Furthermore, to
complete this study, we perform a short data analysis, demonstrating
that cryptocurrencies’ price stability can be disrupted by
technological vulnerabilities characteristic to this market. More
research is needed on this matter, however, with the little data
available, we could show that the bitcoin’s volatility level is
influenced by the number of events targeting it. These evidences show
the implication of cybersecurity risks and Blockchain illiteracy in
the crypto-market players’ behavior. \\
Our results support the general discussion from this literature
survey, while at the same time answer to our initial research
question: ‘Can financial risks be triggered by technological
vulnerabilities of Blockchain technology?’.
The empirical illustration provided in this article cannot be fully
considered as an empirical proof. This is mostly due to the size of
our database. Broadly speaking, information related to crypto-market
is spread all over the internet, making it complicated when it comes
to data collection and research. Up to this point in time, there is no
official or centralized database with attacks performed in the
crypto-market, but rather a collection of mini statistics. On account
of this, our limitation is reducing the possibility to perform
empirical studies and accurately assess certain risks.\\
Finally, we conclude this survey with some research directions, in an
attempt to bridge a part of the existent literature gaps:

\begin{enumerate}
   \item There is need for more research to increase the Blockchain
literacy. In spite of the growing interest in the crypto-market,
practitioners are still challenged to transfer the Blockchain concept
to market-oriented applications. General confidence in this new
technology is often shattered by the negative news, scams or attacks
targeting this market. With their special features and exponential
price changes, cryptocurrencies attract the attention of the large
public, including investors, researchers, regulators or hackers. We
believe that an increased knowledge and understanding about these
innovative technologies will better serve the participants within the
crypto-market in making informed decisions; last but not least, it
will help this market to evolve towards achieving its full potential.
    \item Despite the growing number of empirical papers about
crypto-market, we still lack the theory development in this
field. With our study, we show that using the existing finance
theories is insufficient if the technological characteristics of this
market are not taken into consideration. Blockchain technology is not
just a new tool; it represents a new way of doing business, a new
operational system. Therefore, there is a need of more
cross-disciplinary research, that will take into account the important
functions and implications of this technology (finance, regulation,
cybersecurity, management, etc.).
     \item In recent years, there has been a growing awareness of
climate change and environmental issues. Knowing that PoW
cryptocurrencies represent a threat for our planet health, this
subject needs more attention from both practitioners and
academics. Investors represent an important group of stakeholders in
the crypto-market. Before selecting their preferred investable assets,
investors pay now more attention to their options and generally adopt
the ESG\footnote{Environmental, Social, and Governance
conscientiousness.} evaluation criterion. With the ongoing pandemic
and the continuous expansion of crypto-market, mainly based on PoW
technology, we think that there is an urgent need of research
addressing this challenge.
     \item In the course of the past decade, Blockchain has evolved
while proving its capacity to disrupt various business
sectors. Starting with an already complicated technology, namely
cryptocurrencies, Blockchain development achieved high levels of both
performance and complexity. Innovations such as ICO or
DeFi\footnote{Decentralized finance} projects are built on stacks of
complicated technologies, with each layer carrying an important amount
of (attack) risk. With that in mind, we argue that literature should
address more the vulnerabilities and risks of this market; more
specifically, the ones concerning other Blockchains then bitcoin. An
assessment of the risks and vulnerabilities of the crypto-market as a
whole, could prevent investors from unnecessary loses, diminish the
number of low-quality products and increase performance and efficiency
overall.
     \item As a decentralized system by design, Blockchain technology
is not managed by any central authority but by its own algorithm,
\textit{the code is law}. This leaves the duty of legal and
international regulatory supervision in the hands of the specialists
from governments and industries. The only real progress in this
direction started just in the beginning of 2017
\citep{BotoS2017}. Knowing that a large part of the vulnerabilities
discussed in this survey would have not been possible if proper
regulation was in place, we also consider this an area of further
research.
  \end{enumerate}

Overall, we think this analysis will not only be useful to the
existing participants but also to those considering to enter this
market, them being practitioners or academics. \\


\pagebreak
\bibliography{library}
\bibliographystyle{apacite}
\newpage

\appendix
\section{Appendix}

\begin{table}[ht]
\centering
\caption{\label{reg1}\textbf{Linear regression 1}\\ \footnotesize{\textit{Summary of the OLS regression used to identify the relationship between monthly volatility and the number of events targeting bitcoin.}}}
\setlength{\tabcolsep}{1pt}
\begin{tabulary}{\linewidth}{llllll}
\hline
\textbf{Dep. Variable:}       & \multicolumn{2}{l}{Volatility}        & \textbf{Df Model:}                & \multicolumn{2}{l}{1}             \\
\textbf{Model:}               & \multicolumn{2}{l}{OLS}               & \textbf{R-squared:}               & \multicolumn{2}{l}{0.1938}        \\
\textbf{Method:}              & \multicolumn{2}{l}{Least Squares}     & \textbf{Adj. R-squared:}          & \multicolumn{2}{l}{0.1616}        \\
\textbf{Date:}                & \multicolumn{2}{l}{24 September 2021} & \textbf{F-statistic:}             & \multicolumn{2}{l}{6.01}          \\
\textbf{No. of Observations:} & \multicolumn{2}{l}{27}                & \textbf{Prob. (F-statistic):}     & \multicolumn{2}{l}{0.02156}       \\
\textbf{Df Residuals:}        & \multicolumn{2}{l}{25}                & \textbf{Residual standard error:} & \multicolumn{2}{l}{0.2482}        \\
\hline
\textbf{Coefficients:}   & \multicolumn{5}{l}{}                                                                                          \\
\textbf{Model}           &                  & \textbf{Estimate}    & \multicolumn{1}{c}{\textbf{Std. Error}}           & \textbf{t}        & \multicolumn{1}{c}{\textbf{p-value}}    \\
$H_{1}$                  & (Intercept)      & 0.077                & \multicolumn{1}{c}{0.11783}                       & 0.657             & \multicolumn{1}{c}{0.517} \\
                         & $EVENT_{number}$ & 0.18761              & \multicolumn{1}{c}{0.07653}                       & 2.451             & \multicolumn{1}{c}{0.0216*}  \\
\hline
\end{tabulary}
\\
\footnotesize{\textit{The \textbf{p-value} obtained from this regression is 0.0216*. This result is significant and is less than the significance level alpha: 0.05. We can therefore conclude that there is a relationship between the monthly volatility and the number of events targeting bitcoin.}}
\end{table}



\begin{table}[ht]
\centering
\caption{\label{reg2}\textbf{Linear regression 2}\\ \footnotesize{\textit{Summary of the OLS regression used to identify the relationship between monthly volatility and the amounts lost (BTC).}}}
\setlength{\tabcolsep}{1pt}
\begin{tabulary}{\linewidth}{lllllll}
\hline
\textbf{Dep. Variable:}       & \multicolumn{2}{l}{Volatility}        & \textbf{Df Model:}                      & \multicolumn{2}{l}{1}                       \\
\textbf{Model:}               & \multicolumn{2}{l}{OLS}               & \textbf{R-squared:}                     & \multicolumn{2}{l}{0.007838}                \\
\textbf{Method:}              & \multicolumn{2}{l}{Least Squares}     & \textbf{Adj. R-squared:}                & \multicolumn{2}{l}{-0.03185}                \\
\textbf{Date:}                & \multicolumn{2}{l}{24 September 2021} & \textbf{F-statistic:}                   & \multicolumn{2}{l}{0.1975}                  \\
\textbf{No. of Observations:} & \multicolumn{2}{l}{27}                & \textbf{Prob. (F-statistic):}           & \multicolumn{2}{l}{0.661}                   \\
\textbf{Df Residuals:}        & \multicolumn{2}{l}{25}                & \textbf{Residual standard error:}       & \multicolumn{2}{l}{0.2754}                  \\
\hline
\textbf{Coefficients:}        & \multicolumn{5}{l}{}                                                                                                          \\
\textbf{Model}       &                     & \textbf{Estimate}    & \multicolumn{1}{c}{\textbf{Std. Error}} & \textbf{t} & \multicolumn{1}{c}{\textbf{p-value}} \\
$H_{1}$              & (Intercept)         & 3.34E-01             & \multicolumn{1}{c}{5.57E-02}            & 5.992      & \multicolumn{1}{c}{2.95e-06 ***}    \\
                     & $EVENT_{amount}$    & 1.64E-07             & \multicolumn{1}{c}{3.68E-07}            & 0.444      &\multicolumn{1}{c}{ 0.661} \\
\hline
\end{tabulary}
\footnotesize{\textit{The \textbf{p-value} obtained from this regression is 0.661. This result is higher than the significance level alpha: 0.05. Therefore, we conclude that there is no relationship between the monthly volatility of bitcoin and the amounts lost during the events targeting bitcoin.}}
\end{table}




\begin{table}[ht]
\centering
\caption{\label{events} \textbf{Hacks, thefts and losses events related to Bitcoin} \\ \footnotesize{\textit{Source: \citep{Biais2020}}} }
\small
\setlength{\tabcolsep}{1pt}
\begin{tabular}{c|c|c}
\rowcolor[HTML]{EFEFEF}
\textbf{Date} & \textbf{Amount loss (BTC)} & \textbf{Description}                            \\ \hline
\rowcolor[HTML]{FFFFFF}
6/13/2011     & 25,000                     & Early user Allinvain was   hacked               \\
\rowcolor[HTML]{FFFFFF}
6/19/2011     & 2,000                      & MtGox theft - compromised   account             \\
\rowcolor[HTML]{FFFFFF}
6/25/2011     & 4,019                      & \textit{MyBitcoin theft   - wallet keys hacked} \\
\rowcolor[HTML]{FFFFFF}
7/26/2011     & 17,000                     & \textit{Bitomat loss - Wallet access   lost}    \\
\rowcolor[HTML]{FFFFFF}
7/29/2011 & 78,739 & \textit{MyBitcoin theft - wallet website   hacked}                                \\
\rowcolor[HTML]{FFFFFF}
10/6/2011     & 5,000                      & \textit{Bitcoin7 hack}                          \\
\rowcolor[HTML]{FFFFFF}
10/28/2011    & 2,609                      & \textit{MtGox loss due to hacking}              \\
\rowcolor[HTML]{FFFFFF}
3/1/2012      & 46,653                     & \textit{Linode hacks}                           \\
\rowcolor[HTML]{FFFFFF}
4/13/2012     & 3,171                      & \textit{Betcoin hack}                           \\
\rowcolor[HTML]{FFFFFF}
4/27/2012     & 20,000                     & \textit{Tony76 Silk Road scam}                  \\
\rowcolor[HTML]{FFFFFF}
5/11/2012     & 18,547                     & \textit{Bitcoinica hack}                        \\
\rowcolor[HTML]{FFFFFF}
7/4/2012      & 1,853                      & \textit{MtGox hack}                             \\
\rowcolor[HTML]{FFFFFF}
7/13/2012 & 40,000 & \textit{Bitcoinica theft - due to server   hack}                                  \\
\rowcolor[HTML]{FFFFFF}
7/17/2012     & 180,819                    & \textit{BST Ponzi scheme}                       \\
\rowcolor[HTML]{FFFFFF}
7/31/2012     & 4,500                      & \textit{BTC-e hack}                             \\
\rowcolor[HTML]{FFFFFF}
9/4/2012      & 24,086                     & \textit{Bitfloor theft - wallet keys   hacked}  \\
\rowcolor[HTML]{FFFFFF}
9/28/2012     & 9,222                      & \textit{User Cdecker hacked}                    \\
\rowcolor[HTML]{FFFFFF}
10/17/2012    & 3,500                      & \textit{Trojan horse}                           \\
\rowcolor[HTML]{FFFFFF}
12/21/2012    & 18,787                     & \textit{Bitmarket.eu hack}                      \\
\rowcolor[HTML]{FFFFFF}
5/10/2013     & 1,454                      & \textit{Vircurex hack}                          \\
\rowcolor[HTML]{FFFFFF}
6/10/2013     & 1,300                      & \textit{PicoStocks hack}                        \\
\rowcolor[HTML]{FFFFFF}
10/2/2013     & 29,655                     & \textit{FBI seizes Silk Road funds}             \\
\rowcolor[HTML]{FFFFFF}
10/25/2013    & 144,336                    & \textit{FBI seizes Silk Road funds}             \\
\rowcolor[HTML]{FFFFFF}
10/26/2013    & 22,000                     & \textit{GBL scam}                               \\
\rowcolor[HTML]{FFFFFF}
11/7/2013     & 4,100                      & \textit{Inputs.io hack}                         \\
\rowcolor[HTML]{FFFFFF}
11/12/2013    & 484                        & \textit{Bitcash.cz hack}                        \\
\rowcolor[HTML]{FFFFFF}
11/29/2013    & 5,400                      & \textit{Sheep Marketplace hacked \&   closes}   \\
\rowcolor[HTML]{FFFFFF}
11/29/2013    & 5,896                      & \textit{PicoStocks hack}                        \\
\rowcolor[HTML]{FFFFFF}
2/13/2014     & 4,400                      & \textit{Silk Road 2 hacked}                     \\
\rowcolor[HTML]{FFFFFF}
2/25/2014     & 744,408                    & \textit{MtGox collapse due to hacks   losses}   \\
\rowcolor[HTML]{FFFFFF}
3/4/2014      & 896                        & \textit{Flexcoin hack}                          \\
\rowcolor[HTML]{FFFFFF}
3/4/2014      & 97                         & \textit{Poloniex hack}                          \\
\rowcolor[HTML]{FFFFFF}
3/25/2014     & 950                        & \textit{CryptoRush hacked}                      \\
\rowcolor[HTML]{FFFFFF}
10/14/2014    & 3,894                      & \textit{Mintpal hack}                           \\
\rowcolor[HTML]{FFFFFF}
1/5/2015      & 18,886                     & \textit{Bitstamp hack}                          \\
\rowcolor[HTML]{FFFFFF}
1/28/2015     & 1,000                      & \textit{796Exchange hack}                       \\
\rowcolor[HTML]{FFFFFF}
2/15/2015     & 7,170                      & \textit{BTER hack}                              \\
\rowcolor[HTML]{FFFFFF}
2/17/2015     & 3,000                      & \textit{KipCoin hack}                           \\
\rowcolor[HTML]{FFFFFF}
5/22/2015     & 1,581                      & \textit{Bitfiniex hack}                         \\
\rowcolor[HTML]{FFFFFF}
9/15/2015 & 5,000  & \textit{BitPay phishing scam - hacker   takes over the CEO's accounts and access} \\
\rowcolor[HTML]{FFFFFF}
1/15/2016     & 11,325                     & \textit{Cryptsy hack}                           \\
\rowcolor[HTML]{FFFFFF}
4/7/2016      & 315                        & \textit{ShapeShift hack}                        \\
\rowcolor[HTML]{FFFFFF}
4/13/2016     & 154                        & \textit{ShapeShift hack}                        \\
\rowcolor[HTML]{FFFFFF}
5/14/2016     & 250                        & \textit{Gatecoin hack}                          \\
\rowcolor[HTML]{FFFFFF}
8/2/2016      & 119,756                    & \textit{Bitfinex hack}                          \\
\rowcolor[HTML]{FFFFFF}
10/13/2016    & 2,300                      & \textit{Bitcurex hack}                          \\
\rowcolor[HTML]{FFFFFF}
4/22/2017     & 3,816                      & \textit{Yapizon hack}                           \\
\rowcolor[HTML]{FFFFFF}
7/12/2017     & 1,942                      & \textit{AlphaBay admins assets sized by   FBI}  \\
\rowcolor[HTML]{FFFFFF}
7/20/2017     & 1,200                      & \textit{Hansas funds seized by Dutch   police}  \\
\rowcolor[HTML]{FFFFFF}
12/6/2017     & 4,736                      & \textit{NiceHash hacked}                        \\
\rowcolor[HTML]{FFFFFF}
6/20/2018     & 2,016                      & \textit{Bithumb hacked}                         \\
\rowcolor[HTML]{FFFFFF}
9/20/2018     & 5,966                      & \textit{Zaif hacked}                            \\
\rowcolor[HTML]{FFFFFF}
10/28/2018    & 8                          & \textit{MapleChange hack / scam}               \\
\hline
\end{tabular}
\end{table}



\end{document}