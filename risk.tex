\documentclass{article}


% here all the packages
\usepackage[a4paper, total={6in, 10in}]{geometry}
\usepackage{apacite}
\usepackage[round,sort,colon]{natbib}
\usepackage{booktabs,array, graphicx}
\usepackage{authblk}
\usepackage{multicol, multirow}
\usepackage[table,xcdraw]{xcolor}
\usepackage{setspace}
\usepackage{tabulary,tabularx}
\usepackage[table,xcdraw]{xcolor}
\usepackage{appendix}
\usepackage{amsmath}
\usepackage{url, lscape, geometry, rotating}
\usepackage[utf8]{inputenc}
\usepackage{pdfpages}
\usepackage[skip=0.5ex]{subcaption}
\usepackage[singlelinecheck=false]{caption}

\usepackage{rotating} % Rotating table

\usepackage[colorinlistoftodos]{todonotes}
\newcommand{\TODO}[2][]{\todo[inline,color=blue!10,size=\footnotesize,#1]{\textbf{TODO}: #2}}
\newcommand{\todoJGD}[2][]{\todo[inline,size=\footnotesize,author=JGD,color=orange!50, #1]{#2}}
\usepackage{color,svgcolor}
\usepackage[plainpages=true,verbose]{hyperref}
\makeatletter
\hypersetup{
breaklinks=true,
colorlinks=true,
 linkcolor=blue,
 citecolor=darkgreen,
 urlcolor=darkred,
}
\makeatother

\author[1]{Jean-Guillaume Dumas}
\author[2]{Sonia Jimenez-Garces}
\author[1,2]{Florentina Șoiman \footnote{corresponding author}} 
\affil[1]{Univ. Grenoble Alpes, CNRS, LJK, F-38040 Grenoble, France}
\affil[2]{Univ. Grenoble Alpes, Grenoble INP, CERAG, 38000 Grenoble France}
\affil[ ]{\textit{[Firstname.Lastname]@univ-grenoble-alpes.fr}}

\title{\boldmath Blockchain technology and the crypto-market's risks: A literature survey}

\onehalfspacing
\begin{document}
\maketitle
\author{}

\begin{abstract}
This paper provides a literature survey on the vulnerabilities and risks of Blockchain technology and the crypto-market. Since their creation, the crypto-market and Blockchain technology are still very much challenged and far from mainstream adoption. Thus, we propose a detailed literature survey focusing on the relationship between technological characteristics and financial risks. Furthermore, to complete this study, we propose ways to determine the likelihood of technological vulnerabilities triggering financial risks. Additionally, we find a significant relationship
between Blockchain attacks and cryptocurrency volatility, illustrating the relationship between technological vulnerabilities and financial risk. While we consider the existent literature on the vulnerabilities and risks associated with the crypto-market, we cannot ignore the little work concerning crypto-fundamentals. Alleviating this gap will require more theoretical models and empirical research that encompass the technological characteristics of this market. Our contributions are threefold. First, we perform a literature survey comprising the crypto-market’s risks. Secondly, we show a link between technological risks and financial ones. Thirdly, we provide empirical results showing that bitcoin’s price stability is disturbed by technological vulnerabilities.
\\
\noindent \textbf{Keywords: Blockchain, Risk assessment, Financial risks, Attacks, Literature survey }
\\
\noindent \textbf{JEL Codes: G10, G15, G19}
\end{abstract}

\pagebreak
\section{Introduction}

Everyone has heard about the enormous potential of the Blockchain
technology and the fact that it might revolutionize business models
and reinvent the contemporary firms and economies. At the same time,
we know that it is still far from keeping all its promises, and before
that happens, Blockchain has first to overcome its technological,
organizational and social barriers \citep{Iansiti2017, Charles2019}. As mentioned by \cite{Beddiar2018}, “The Internet has democratized the information, the Blockchain will democratize the
transaction” \footnote{Own translation from original: ” Internet a
  démocratisé l’information, la Blockchain va démocratiser la
  transaction” \citep{Charles2019}}; however, there is still a lot of
work left for research before that happens and much experience to
gain before the technology will mature \citep{Charles2019}. This
global distributed, open and transparent database, which stores and
transfers information of any kind (money, art, science, titles, votes,
etc.) has the potential to create new foundations for the economy and
business sector. Blockchain might be a complex technology, but the concept behind it is quite simple \citep{Tapscott2016, Iansiti2017}.


Inspired by the existing systems and technologies, the solutions
promised by Blockchain seem to be far beyond what we have already
seen. Little by little, Blockchain is taking over many sectors of the
economy, and a growing number of organizations are declaring their
enthusiasm and interest in using it \citep{Collomb2016}. Given the spread of Blockchain-based solutions across various industries and the growing interest in using them, there is a need for researchers
and market participants to gain an understanding of what it means to be
part of the crypto-market.

As previously mentioned, Blockchain needs to overcome a series of
challenges before becoming a mainstream technology
\citep{Wachsman2019}. According to \cite{Iansiti2017}, two
dimensions are affecting the way technology evolves. The first dimension
represents novelty, referring to the degree of originality and
uniqueness compared to the existing systems. This dimension
also implies the difficulty in seeing the use and innovation of
technology. The second dimension refers to complexity, implying the
extent to which this technology touches various fields, regardless of the market or area of expertise \citep{Iansiti2017, Notheisen2019}. The
same idea is sustained in the surveys conducted by Deloitte and Underscore companies. While assessing the Blockchain adoption, Deloitte found out that some of the main barriers are: technological complexity, regulatory issues, lack of in-house
skills and understanding, security threats, and the uncertain
profitability \citep{Pawczuk2019, UnderscoreVC2018}. In 2018,
\cite{Gazali2018} explored the relationship between human conduct and
the intention to invest in the crypto-market. Consequently, they found
out that the attitude towards the crypto-market, the social norms
\footnote{decisions are made based on the actual trends and influenced
  by a mentality such like: “if I lose, at least I am not alone”)},
the risk tolerance and the perceived benefits coming from using this
technology, represent some of the main factors influencing the
interested parties to invest or be part of the crypto-market.


Regardless of the high potential and great innovative solutions brought by
Blockchain this technology gained most of its fame thanks to its
vulnerabilities. The cryptocurrencies’ volatility and the numerous
cyber-attacks suffered by this technology represent the main driving
factors towards the Blockchain’s popularity. Among the existing research literature, several studies have addressed the crypto-market risks. Some to find solutions to these vulnerabilities \citep{Bonneau2015, Stewart2018, Ma2018, Goffard2019, Morganti2019,   Drljevic2019, Patel2020}, while others just to increase general awareness \citep{Saad2019, Canh2019, Lemieux2016, Gazali2018,
  Lu2019a}.


In previous papers, risks are usually treated independently based on their nature (i.e., economic, political, regulatory, etc.). Following the review of the existing research, we propose to fill the literature gap and perform an analysis in parallel of both financial and technological risks. Our contribution shows that these risks, regardless of their nature, have many characteristics in common. Moreover, we offer ways to determine the likelihood that technological risks could transform into financial ones and provide a short empirical demonstration.


This study is a literature-based research. Compared to other areas, finance is mostly dominated by quantitative analyses. Of the same mind as \cite{Corbet2019} our study follows a similar belief, namely: “for new research areas such as those based around cryptocurrencies, a literature analysis can be the most powerful tool to inform academics, professionals, and policy-makers about the current state of knowledge, consensuses, and ambiguities in the emerging discipline.” In conducting this research, we have used various types of information, from both academic \footnote{Academic journals, academic theses} and non-academic \footnote{Websites, official reports issued by research or governmental organizations, magazines, etc.} literature. The selection of papers was performed by first taking into account the topic of investigation; afterward, information was grouped by type of risk. In our search, we have used many keywords such as: crypto, Blockchain, financial risk, technological risk, attack, financial behavior, Blockchain literacy, etc. The contributions proceeding from this literature survey answer our research question: ‘Can financial risks be triggered by technological vulnerabilities of Blockchain technology?’. We demonstrate that cryptocurrencies’ price stability can be disrupted by technological vulnerabilities characteristic of this market.

To enlighten our research problem, the objective of this survey is to provide a two-dimension risk analysis (technological and financial) completed by an assessment of triggering elements (the
likelihood). Furthermore, following the example of \cite{Benoit2017}
literature survey, we complete this work with a short data analysis. In line with the statements made in the literature review, we show that bitcoin’s price instability (financial risk) can be triggered by attacks targeting the crypto-market (technological vulnerability).


This paper is organized as follows. Section 2 presents the assessment of technological and financial risks; Section 3 proposes a brief empirical illustration. Section 4 discusses the results and concludes.



\section{Blockchain risks assessment}

In this section, we perform a theoretical risk assessment of the
crypto-market. The goals of this assessment are:  \\

 \begin{tabular}{@{$\bullet$ }ll}
&To understand the vulnerabilities of Blockchain and their possible consequences and impact; \\
&To offer a broad view on possible financial and technological risks for Blockchain stakeholders .\\
\end{tabular} \\ \\
According to \cite{Leemoon2017}, crypto-market’s challenges can be divided into four main areas:
\begin{enumerate}
  \item Technological issues
  \item Financial issues
  \item Policy and legal issues
  \item Political issues
\end{enumerate}

While all four types of risks are indisputably affecting the crypto-market development and slowing its acceptance, we consider that the first two could represent a starting point and reliable support in designing a better legal framework. That being said, in this study, we tackle the first two categories, leaving the last two for future research. We make a parallel analysis between the technological and financial risks. 


The complexity of this technology, inherited by nature, represents a challenge for users, investors, and any other participants from this
market \citep{Salmela2019}. Highly secure at first sight, Blockchain
is not exempt from risks but is instead an imperfect innovation leaving
generous room for many improvements \citep{Iwamura2019}. According to \cite{Swan2017} Blockchain technology is the only one that has the
potential to change or, better said, to revolutionize the way
businesses and financial markets work.


According to the latest surveys performed, the main barriers slowing down the Blockchain’s adoption are: scalability issues, insufficient regulation, the unproven or debatable value of technology, security threats, lack of in-house skills, and uncertain rate of return \citep{UnderscoreVC2018, Pawczuk2019}. As we can observe, most of the mentioned obstacles are either technological or finance-related. These findings encourage us to perform a risk assessment and support the necessity of prioritizing the first two categories of risks, namely the financial and technological ones.



\subsection{Technological risks}
We here systemize the crypto-market threats in accordance with their
nature, namely, consensus-level attacks, network-level attacks,
cryptographic key attacks, and smart contract attacks. There are many
types of attacks that are not discussed in this study. However, we
tried to cover the most important ones by taking into account the
likelihood, the exposure of the crypto-market to such incidents, and
the (financial) impact they might have.


\textbf{Consensus algorithms} for Blockchain technology represent a
code-based protocol, aiming to facilitate reaching agreement processes
within a network. These algorithms came as a solution to the
“Byzantine General Problem”, which concerns the failure of reaching
consensus due to faulty actors \citep{Zhang2019}. The most popular and
widespread consensus algorithms in Blockchain technology are the
Proof of Work (PoW), Proof of Stake (PoS), and the Practical Byzantine
Fault Tolerance (PBFT) protocols (see Table \ref{consensus}).



\begin{table}[h!]
\centering
\caption{\textbf{Comparison of most notable consensus mechanisms used in the Blockchain applications}}
\label{consensus}
\begin{tabular}{||c c c c||}
 \hline
\textbf{Proprieties} & \textbf{PoW} & \textbf{PoS} & \textbf{PBFT}  \\ [0.5ex]
 \hline
 \textbf{Blockchain type} & \textit{Permissionless} & \textit{Permissionless} & \textit{Permissioned} \\
\hline
  \textbf{Fault Tolerance} & \textit{$<$50\%(of computing power)} & \textit{$<$50\%(of stake)} & \textit{$<$33\%(of faulty nodes)} \\ [1ex]
 \hline
\end{tabular}

\end{table}

The most noteworthy \textbf{attacks at the consensus level}, are:

\underline{Nothing at stake attack:} on the PoS protocol, where low
stake owners try to decrease the value of cryptocurrency. Indeed, the
control inside the system is given based on the user’s wealth,
potentially combined with other factors (coin age-based selection or
random factors). Any PoS Blockchain can be exposed to this type of
attack, especially in their beginnings, when there are no real
imbalances among the users’ wealth and low stake owners will not lose
much \citep{Morganti2019}.


\underline{The majority attack ($>$50\% attack):} means that the consensus protocol is compromised, functioning as a monopolistic
system. Considering its possible implications, the majority attack is also considered a security issue. Moreover, considering the
target type, it can be split into two variants: “the $>$50\% (or 51\%)
computational power attack” \footnote{an attack on the PoW protocol,
  implying the possession of more than 50\% of the total mining power,
  with the purpose to manipulate and corrupt the network} and “The
51\% stake attack” \footnote{An attack targeting the PoS protocol; it
  implies the possession of more than 50\% of the total circulating
  supply of coins (within the same network) with the purpose to gain
  monopoly power and mislead the system for profit purposes. It is
  conceptually similar to computational power attack.}
\citep{Tuwiner2021, Blockchain.com2020}.


Bitcoin has never experienced a successful majority attack. However,
we cannot say the same about altcoins: Feathercoin (June 2013),
Bitcoin Gold (May 2018), Vertcoin (December 2018), Ethereum Classic
(January 2019) and Bitcoin Cash (May 2019) \citep{Beigel2019}. The size of the Blockchain network very much influences the difficulty of executing an attack. Table \ref{attacks} shows how expensive it is to perform a majority attack, depending on the cryptocurrency. These costs are computed taking into account the expenses incurred in the mining process, namely the network hash rate \& the Nicehash cost in
BTC /per hour (rented PC power). These values can change every minute,
as the cryptocurrencies prices have a strong influence \citep{Crypto51.app2020}.


\begin{table}[h!]
\centering
\caption{\textbf{PoW 51\% attack cost for the top 7 cryptocurrencies.}}
\begin{tabular}{||c c c ||}
 \hline
\textbf{System} & \textbf{Hash rate \footnote{The Hashing power is expressed in different units: TH = TeraHash; MH = MegaHash; KH = KiloHash; GH = GigaHash \citep{CoinGuides.org2020}.}} & \textbf{1 h attack estimated Cost}  \\ [0.5ex]
 \hline
 \textbf{Bitcoin} & \textit{114,915 PH/s} & \textit{\$716,072} \\
\hline
  \textbf{Ethereum} & \textit{253 TH/s} & \textit{\$418,438} \\
\hline
  \textbf{Litecoin} & \textit{227 TH/s} & \textit{\$29,287)} \\
\hline
  \textbf{B. Cash} & \textit{1,374 PH/s} & \textit{\$8,560)} \\
\hline
  \textbf{Zcash} & \textit{8 GH/s} & \textit{\$8,710} \\
\hline
  \textbf{B. SV} & \textit{1,109 PH/s)} & \textit{\$6,912} \\
\hline
  \textbf{Dash} & \textit{7 PH/s)} & \textit{\$3,246}  \\ [1ex]
\hline
\end{tabular}
 \\ \footnotesize{\textit{Values computed as per 10th February 2021}} \\
Source: derived from \cite{Crypto51.app2020}
\label{attacks}
\end{table}


\textbf{Network level attacks} are widely considered difficult and
expensive to perform \citep{Koshik2019}; however, they should never be
regarded as impossible.


\underline{DDoS (Distributed Denial of Service:} refers to an attack
on the host, aiming to disrupt the normal operation process. If, for
example, the (host) Blockchain system is under attack, it can become
unresponsive, unavailable. The system is compromised by being feed
with misleading information or large amounts of data
\citep{Zhang2019}. DDoS attacks can have a notable impact within the
crypto-market, as they can target Blockchains \footnote{The difficulty to execute an attack is very much influenced by the size of the Blockchain network. Private Blockchains are considered more exposed compared to the public ones, as they usually grow around just 100 nodes. The adversary needs to control only 33\% of the network to perform an attack, which is easier to achieve in small Blockchains \citep{Saad2019}.}, exchange and trading platforms, and even mining pools \citep{Abhishta2019, Litecoinpool.org2020}. These attacks are
highly associated with the increase in value and popularity of the
cryptocurrencies \citep{Crothers2021}.


Some other notable examples of network-level attacks, worth to mention
if we take into account the exposure and powerful impact they could
have, are the Sybil attack \footnote{a user creates multiple
  identities and uses them to gain dominance and manipulate the
  Blockchain system} and the Eclipse attack \footnote{similar to a
  Sybil attack, Eclipse misleads its victims such as they will see and
  believe a different truth than the rest of the network}. From our
knowledge, there is no Sybil or Eclipse attack successfully performed
on the Blockchain technology, in practice, but researchers have made
theoretical demonstrations for the Eclipse attacks on both PoW (Ether
and Bitcoin) \citep{Heilman2015,Packtpub2019, Marcus2018, Wust2016}
and PoS networks \citep{Zhang2019}. Usually, the network-level attacks
are planned so they can precede other assaults
\citep{Morganti2019}. \\


\textbf{Cryptographic key attacks.} In Blockchain technology,
cryptographic keys give access to funds (through crypto wallets) and
play a critical role in transactional processes. In other words,
anyone handling the cryptographic keys can access the wallet account
and freely manage the associated funds. These keys are stored in
crypto wallets. According to the version of crypto wallet used
(software, hardware, cloud, brain\footnote{It is a type of wallet which
  gives the user the option to generate a key using a password (a
  word, number, combination of both, etc.). This type of wallet and
  keys are considered weak in terms of security.} or paper), the keys
are more or less safe (hardware \& paper - most secure, software,
brain \& cloudless secure). Having such a variety of key storage
options gives attackers ideas to approach the wallets in different
ways.


\underline{Wallet attack:} The main causes behind wallet attacks are
system hacking, software vulnerabilities, malware, or incorrect usage
from the users’ side. The objective is to obtain (steal) the private
key, with which the attacker can mislead the system, perform
unauthorized transactions, and steal coins (send them into the thief’s
wallet using the victim’s private key). Compared to any other type of
crypto attacks, the ones targeting the wallets are among the most
common and harmful incidents \footnote{In 2018, Coincheck’s wallets
  were hacked and lost \$530 million worth of NEM. This incident
  surpasses even the losses of the Mt. Gox case, being classified as the most significant theft in the crypto history \citep{Shane2018}.}. This statement is also supported by the Blockchain Graveyard organization, as according to their thorough analysis on the incidents associated with
Blockchain, more than half relate to wallet attacks
\citep{Magoo.github.io2020}.


Some other notable examples of attacks at this level are: the
\underline{Random number generator attack} \footnote{targets the weak
  security of the cryptographic keys due to insufficient randomness
  used in their generation process, making them easy to predict
  (Independent Security Evaluators, 2019); despite the common knowledge that the cryptographic keys are difficult to break,
  a combination of weak hashing algorithms and skilled
  hackers have led to such kind of incidents.} and \underline{Quantum
  attacks} \footnote{performed with the quantum computers (QC); In the
  context of Blockchain, they can break the cryptographic keys,
  corrupt the hashing functions and forge digital signatures. These
  attacks can have serious implications for the Blockchain network,
  implying theft of the users’ funds, crypto wallets corruption,
  dominance over the network and even possible recreation of the
  entire Blockchain. It is maybe a matter of time until we will have
  a QC powerful enough able to break the Blockchain technology
  \citep{Fernandez-Carames2020, Stewart2018}.} . \\


\textbf{Smart contract attacks} mainly refer to the manipulation of
external data entered in the Blockchain (through oracle technology),
misleading the execution of the smart contract. The trigger represents
information related to external events, which affects the contract’s
conditions. The information is manually introduced, reason why, the
execution of the system can be easily misled. Blockchain is an
open-source technology, giving access to its full code. This is an
opportunity for intruders, who may take advantage of this feature and
exploit it with malevolent intentions. Concurrently, if the
programming language used in the smart contract has weaknesses, this
might also create the perfect opportunity for any hacker to initiate a
successful attack \citep{Hasanova2019, Atzei2017}.


\underline{Re-entrancy attack}, as a variant, refers to a malfunction
in the smart contract protocol. During the attack, the hacker is
sending multiple requests to the system, as for example, invoking the
call function continuously until the gas supply ends. Overwhelmed by
the avalanche of orders, the system will perform inaccurately
\citep{Hasanova2019}.



A summary of all technological risks discussed above will be presented in Table \ref{tec}.


\newcommand{\spheading}[2][10em]{% \spheading[<width>]{<stuff>}
  \rotatebox{90}{\parbox{#1}{\raggedright #2}}}
\newcommand{\floatfootnote}[1]{\ifx\[$\else\footnote{#1}\fi} %$

%\begin{landscape}
\begin{sidewaystable}[htbp]
\centering
\small
\caption{\label{tec}\textbf{Summary of technological risks}}
%\resizebox{1.65\textwidth}{!}{
\begin{tabulary}{\linewidth}{p{25pt}|L|p{200pt}p{175pt}}
%\begin{tabularx}{\textwidth}{|c|lXX}
\hline
 &
  \centering{\textbf{Risk}} &
   \centering{\textbf{Consequences}} &
   {\textbf{Exposure}}  \\
\hline
\rowcolor[HTML]{FFFFFF}
{\multirow{7}{*}{\rotatebox[origin=r]{90}{\cellcolor[HTML]{FFFFFF}\textbf{Consensus level attack}}}} &
 {{\cellcolor[HTML]{EFEFEF}\textbf{Nothing   at stake attack}}} &
 {\cellcolor[HTML]{EFEFEF}\textit{· Manipulates the system by entering
     invalid data  \mbox{· Monopolized consensus process}}} &
  {{\cellcolor[HTML]{EFEFEF}\textit{Blockchains using PoS (over 350 cryptocurrencies\floatfootnote{The total number of cryptocurrencies is over 2200 \citep{CryptoSlate.com2020}}) source:\citep{CryptoSlate.com2020}}}}
 \\
&
  {\cellcolor[HTML]{FFFFFF}} &
  {\cellcolor[HTML]{FFFFFF}\textit{·  Manipulates the system}} &
  {\cellcolor[HTML]{FFFFFF}\textit{Blockchains using PoW consensus (over 500 cryptocurrencies);}} \\
 &
  {\cellcolor[HTML]{FFFFFF}} &
  {\cellcolor[HTML]{FFFFFF}\textit{·  Monopolized consensus process}} &
  {\cellcolor[HTML]{FFFFFF}\textit{Blockchains using PoS (over 350 cryptocurrencies)source: \citep{CryptoSlate.com2020}}} \\
 &
  {\cellcolor[HTML]{FFFFFF}} &
  {\cellcolor[HTML]{FFFFFF}\textit{· Enters invalid data in the system}} &
  {\cellcolor[HTML]{FFFFFF}} \\
 &
  {\cellcolor[HTML]{FFFFFF}} &
  {\cellcolor[HTML]{FFFFFF}\textit{· Forks the Blockchain}} &
  {\multirow{-2}{*}{\cellcolor[HTML]{FFFFFF}\textit{Mining pools\floatfootnote{The total number of mining pools is not known, as there are many which keep their identity secret. Therefore, we cannot accurately assess the market exposure with this respect.}}}} \\
 &
  {\multirow{-5}{*}{\cellcolor[HTML]{FFFFFF}\textbf{Majority attack}}} &
  {\cellcolor[HTML]{FFFFFF}\textit{·          Performs other attacks (Eclipse, double spending, DoS)}} &
  {\cellcolor[HTML]{FFFFFF}\textit{}} \\ \hline
&
  {\cellcolor[HTML]{EFEFEF}} &
  {\cellcolor[HTML]{EFEFEF}\textit{·            Manipulates the system by entering invalid or large flow of data}} &
  {\cellcolor[HTML]{EFEFEF}\textit{All Blockchains (small ones most exposed)}} \\
\rowcolor[HTML]{EFEFEF}
{\cellcolor[HTML]{FFFFFF}} &
  {\cellcolor[HTML]{EFEFEF}} &
  {\cellcolor[HTML]{EFEFEF}\textit{·          Disrupts the normal operation process}} &
  {\cellcolor[HTML]{EFEFEF}\textit{Mining pools}} \\
 &
  {\multirow{-3}{*}{\cellcolor[HTML]{EFEFEF}\textbf{DDoS   attack}}} &
  {\cellcolor[HTML]{EFEFEF}\textit{·          Knocks out part of or the whole network}} &
  {\cellcolor[HTML]{EFEFEF}\textit{Exchange platforms}} \\
\rowcolor[HTML]{FFFFFF}
{\cellcolor[HTML]{FFFFFF}} &
  {\cellcolor[HTML]{FFFFFF}} &
  {\cellcolor[HTML]{FFFFFF}\textit{·            Manipulates the system}} &
  {\cellcolor[HTML]{FFFFFF}} \\
\rowcolor[HTML]{FFFFFF}
{\cellcolor[HTML]{FFFFFF}} &
  {\cellcolor[HTML]{FFFFFF}} &
  {\cellcolor[HTML]{FFFFFF}\textit{·          Monopolized consensus process}} &
  {\cellcolor[HTML]{FFFFFF}} \\
\rowcolor[HTML]{FFFFFF}
{\cellcolor[HTML]{FFFFFF}} &
  {\multirow{-3}{*}{\cellcolor[HTML]{FFFFFF}\textbf{Sybil   attack}}} &
  {\cellcolor[HTML]{FFFFFF}\textit{·          Enters invalid data in the system}} &
  {\multirow{-3}{*}{\cellcolor[HTML]{FFFFFF}\textit{Permissionless   Blockchains}}} \\
\rowcolor[HTML]{EFEFEF}
{\cellcolor[HTML]{FFFFFF}} &
  {\cellcolor[HTML]{EFEFEF}} &
  {\cellcolor[HTML]{EFEFEF}\textit{·            Manipulates the system}} &
  {\cellcolor[HTML]{EFEFEF}} \\
\rowcolor[HTML]{EFEFEF}
{\cellcolor[HTML]{FFFFFF}} &
  {\cellcolor[HTML]{EFEFEF}} &
  {\cellcolor[HTML]{EFEFEF}\textit{·          Monopolized consensus process}} &
  {\cellcolor[HTML]{EFEFEF}} \\
\rowcolor[HTML]{EFEFEF}
{\multirow{-9}{*}{\rotatebox[origin=c]{90}{\cellcolor[HTML]{FFFFFF}\textbf{Network level attack}}}} &
  {\multirow{-3}{*}{\cellcolor[HTML]{EFEFEF}\textbf{Eclipse attack}}} &
  {\cellcolor[HTML]{EFEFEF}\textit{·          Enters invalid data in the system}} &
  {\multirow{-3}{*}{\cellcolor[HTML]{EFEFEF}\textit{Permissionless   Blockchains}}} \\ \hline
\rowcolor[HTML]{FFFFFF}
{\cellcolor[HTML]{FFFFFF}} &
  {\cellcolor[HTML]{FFFFFF}} &
  {\cellcolor[HTML]{FFFFFF}\textit{·            Steals the cryptographic keys}} &
  {\cellcolor[HTML]{FFFFFF}} \\
\rowcolor[HTML]{FFFFFF}
{\cellcolor[HTML]{FFFFFF}} &
  {\cellcolor[HTML]{FFFFFF}} &
  {\cellcolor[HTML]{FFFFFF}\textit{·          Takes the control of the afferent funds}} &
  {\cellcolor[HTML]{FFFFFF}} \\
\rowcolor[HTML]{FFFFFF}
{\multirow{7}{*}{\cellcolor[HTML]{FFFFFF}\spheading[50pt]{\textbf{Cryptographic key threats}}}} &
  {\multirow{-3}{*}{\cellcolor[HTML]{FFFFFF}\textbf{Wallet attack}}} &
  {\cellcolor[HTML]{FFFFFF}\textit{·          Deters the security and trust of the users}} &
  {\multirow{-3}{*}{\cellcolor[HTML]{FFFFFF}\textit{All Blockchains}}} \\
 &
  {\cellcolor[HTML]{EFEFEF}\textbf{Random   number generator attack}} &
  {\cellcolor[HTML]{EFEFEF}\textit{·  Corrupts the cryptographic keys \& crypto wallets}} &
  {\cellcolor[HTML]{EFEFEF}\textit{All Blockchains}} \\
 &
  {\cellcolor[HTML]{FFFFFF}} &
  {\cellcolor[HTML]{FFFFFF}\textit{·            Corrupts the cryptographic keys \& crypto   wallets}} &
  {\cellcolor[HTML]{FFFFFF}} \\
 &
  {\cellcolor[HTML]{FFFFFF}} &
  {\cellcolor[HTML]{FFFFFF}\textit{·          Forges hashing functions \& digital signatures}} &
  {\cellcolor[HTML]{FFFFFF}} \\
&
  {\multirow{-3}{*}{\cellcolor[HTML]{FFFFFF}\textbf{Quantum attacks}}} &
  {\cellcolor[HTML]{FFFFFF}\textit{·          Rewrites Blockchain and manipulation of the network}} &
  {\multirow{-3}{*}{\cellcolor[HTML]{FFFFFF}\textit{All Blockchains}}} \\ \hline
\rowcolor[HTML]{EFEFEF}
{\multirow{2}{*}{\cellcolor[HTML]{FFFFFF}\spheading[50pt]{\centering\textbf{Smart
        contract threats}}}}
 &
  {\cellcolor[HTML]{EFEFEF}\textbf{Reentrancy   attack}} &
  {\cellcolor[HTML]{EFEFEF}\textit{Manipulates the network   \& spends unlimited}} &
  {\cellcolor[HTML]{EFEFEF}\textit{Blockchains   supporting smart contracts (over 50 cryptocurrencies) Source:\citep{CryptoSlate.com2020}}} \\
 &
  {\cellcolor[HTML]{FFFFFF}\textbf{Smart contract attack}} &
  {\cellcolor[HTML]{FFFFFF}\textit{Misleads the technology’s application}} &
  {\cellcolor[HTML]{FFFFFF}\textit{Blockchains supporting smart contracts (over 50 cryptocurrencies) Source:\citep{CryptoSlate.com2020}}} \\ \hline
\end{tabulary}
%}
\end{sidewaystable}

%\end{landscape}





\subsection{Financial risks}

 In this section, we give the example of several financial risks that can
 be triggered by technological risks. After detailing how this phenomenon happens and in what kind of circumstances, we propose a
 conceptual metric with the purpose of emphasizing the likelihood that
 these technological risks may transform into financial ones.


\underline{Determining the likelihood:} The likelihood that the
technological risks may transform into financial risks can be
established by taking into account the severity \footnote{Financial
  loses and investment cost incurred.} effect and probability of
occurrence of triggering elements. Here, we will also introduce the
concepts of financial behavior, responsible investment, and Blockchain
literacy as possible tools for assessing risk. Measurement plays an essential role in management. Up to this point, we have different
tools to measure financial risks; however, things are not as simple
when talking about the triggering elements. According to
\cite{Kaplan1992}, if we cannot measure something, then we cannot
properly manage it. Therefore, in this part of the assessment, we
propose ways to measure the probability of technological
vulnerabilities triggering financial risk. \\


\textbf{Total market risk.} This is the financial risk arising from
high movement in market prices. The most used measure for appraising
the total market risk of an asset is the volatility of its market
returns. Following the traditional financial theory, the total market
risk can be decomposed into the systematic risk and the specific
one. If the crypto-market is vulnerable to a risk threatening the whole
market, this could be a systematic risk. On the other hand, if we
consider risks targeting a specific crypto-asset or type of
Blockchain, then this could be an example of specific risk
\footnote{Specific risk concerns isolated cases (one crypto-asset or a
  specific group, usually not dominating the market) and has fewer
  casualties than a systematic risk, which affects a large part of the
  market or the whole.}.



 From the previous list, by taking into consideration the (technological) risks’ exposure
 and their consequential power, we can quickly identify several
 attacks capable of triggering financial risks. For instance, the majority
 attacks (exposure: almost half of the total crypto-market, plus the mining pools), Sybil and Eclipse attacks (target: Permissionless Blockchains - the most common and significant representatives of this market-), DDoS attack, wallet attack, random number generator attack, and quantum attacks (target: all types of Blockchain) can be considered potential triggers for systematic risk. At the same time, if affecting just one type of Blockchain, one cryptocurrency, or a few casualties such as a mining pool/exchange platform, the same technological risk can trigger a specific one.


 It is well known that regulatory and cybersecurity-related events influence the crypto-assets prices \citep{Corbet2019}. Subsequently, such events influence the investors’ behavior, impacting the crypto-market’s volatility. It was also proved that cryptocurrencies suffer from contagion effects (herding behavior)\citep{DaGamaSilva2019}. Bitcoin, Ether, or any other strong and well-known currency have proved their influence over the evolution of the whole cryptocurrency market. In 2017, when Bitcoin prices skyrocketed and crashed, the rest of the cryptocurrencies followed a similar trend \citep{Antonakakis2019, Pereira2019}. The strong power of influence and the herding behavior present in the crypto-market may trigger systematic risk. Here, we have the perfect example of how an independent event, initially affecting one currency (specific risk), can eventually transform into a systematic risk \footnote{this was possible through investors’ behavior, which tends to associate Bitcoin’s image with the one of the whole market.}, impacting the whole market \citep{Jain2019}. It is well known that systematic risk can be triggered by various factors such as socio-political, economic, and any other market-related events. In the crypto-market, we can see that on top of the already existing factors, we also have technological vulnerabilities as a possible trigger. \cite{Koutmos2020} showed that despite Bitcoin’s relative independent price behavior, it is still exposed to the same market risks as conventional financial assets. Under the hypothesis of traditional financial theory, the specific risk is diversifiable and is not priced by the market. On the opposite, investors require a risk premium, and thus, higher returns for compensating the systematic risk they incur. 
 
 Finally, we state that in spite of its technological nature and distinct vulnerabilities, the whole crypto-market, similar to the traditional financial market, is susceptible to the same financial risks, namely systematic and specific risks. 


\underline{Likelihood:} The main triggers for market risks are 
cyber-attacks (technological risks). According to the Blockchain-Graveyard database of crypto attacks, the most frequent and damaging are the ones on cryptographic keys (about half of the total incidents), followed by application vulnerabilities (security breaches) and protocol issues \citep{Magoo.github.io2020}. Like a vicious circle, good financial conditions in the crypto-market can motivate intruders to perform more attacks \citep{Crothers2021}. Eventually, depending on the amplitude of
damaged caused, technological risks might transpose into different
financial risks. Since attacks are pretty common in the crypto-market
and usually imply important financial losses, we state that the
likelihood is high. \\


 \textbf{Information risk} risk refers to the imbalance of information
 spread among the market players. Conceptually speaking, thanks to its
 features, Blockchain technology represents itself a valuable tool in
 reducing information asymmetry, assuring transparency and
 trust. However, along with the evolution of the crypto-market, these
 innovations became more complex, challenging investors and users to
 acknowledge the potential. The novelty and technical nature of
 the crypto-market may get stakeholders into trouble, as some do not
 understand it. At the same time, the lack of knowledge and specific
 skills, sometimes completed by the insufficient information supplied
 to the public, increases the uncertainty and restrain towards the
 whole market.


Compared to any other Blockchain application, initial Coin Offerings (ICO) impose most of the transparency and information asymmetry problems. The complexity of ICOs’ white paper \footnote{a document describing the technology used in the Blockchain project (ICO). It has the purpose to convince the public that the new
  crypto-asset offers a good investment opportunity.}, investors’ lack of training and insufficient regulation led to manipulation and financial losses for investors. According to the existing literature, most investors in this market lack the required capabilities to interpret the market’s signals. The discrepancy between the traditional market and crypto-market pushes investors and users towards questionable sources of information such as social media. Here, the selection is based on the ‘easy-to-interpret’ criteria rather than quality and credibility. At the same time, the general opinion surrounding the crypto-market seems to influence the players (investors and users), which might take decisions rather based on the social trends (led by a herd mentality \footnote{an “If I am losing, at least I am not losing alone’ mentality – investors might believe that following trends or the majority provides some security and makes losses easier to tolerate \citep{Gazali2018}}) than rationally. In line with our arguments, \cite{Florysiak2018} states that in comparison to an IPO prospectus, the information shared through ICO white paper is less standardized (due to insufficient regulation) and more complicated to understand since it describes a new concept of technology business; therefore this information is often omitted by investors or other professionals part of this market. Moreover, the authors discuss that the ICO expert ratings are uncorrelated to the content of the white paper, meaning that ratings do not accurately reflect the quality of the project or technology \citep{Florysiak2018} and which eventually will make it more complicated to integrate information within the market. This could explain the inefficiency of the crypto-market, despite the quantity of information available \citep{RuiChen2020, Gazali2018}.


\underline{Likelihood:} Among the most important factors responsible for information risk in the crypto-market, we have the lack of available information (e.g., white/yellow papers, inconsistent data) and insufficient knowledge or understanding for investors and users. Due to the poor regulatory framework, intruders found an opportunity to become rich overnight. They issue low-quality crypto-assets, about which there is little information available (incomplete white papers or inconsistent data), and use them to trick the other market players. This risk is behind most of the fraudulent coins or low-quality ICO projects. Reputation might attract more enthusiasts in this market; therefore, we believe that the investors interested in cryptos are pretty various. Here, we introduce Blockchain literacy (the ability to understand the Blockchain related knowledge and make informed and effective decisions \citep{VanRooij2011})and financial behavior (how individuals gather and interpret information, eventually reflecting in decisional processes \citep{DeBondt2008}), concepts, as essential factors in the way the market evolves \citep{Zhao2021}. Market signals can be complex, including both information and noise \citep{Rizzi2008}.

Less mysterious than at the beginning, however, still significantly complicated, the Blockchain world might pose some problems in understanding. Blockchain illiteracy leads to irrational behavior, which eventually reflects in inefficient markets. Taking into account the large number of crypto scams and the important financial losses incurred (especially during the Bitcoin bubble 2017-2018 \citep{Zetzsche2019, Liebau2019}), we state that the likelihood for this risk is high. \\


\textbf{Liquidity risk.} A market is said to be liquid if an agent can rapidly make some significant trades without creating an important change in the price (small market impact). In other words, in a liquid market, transactions will likely not change the price, but new information will be smoothly incorporated. On the other hand, an illiquid market (often linked to an inefficient market) will reflect in large volatility in prices (hence a higher probability of an unfair price), a lower number of investors, and lower chances to transact/trade. Liquidity risk can be split into three categories: assets liquidity (refers to the interaction between sellers and buyers on the platform and the asset availability on exchanges), exchange liquidity (refers to the interaction between makers and takers concerning the assets’ and the orders’ supply) and market liquidity (encompasses the first two) \citep{Crowell2020}. According to \cite{Corbet2019}, liquidity risk is also highly correlated with the events concerning cyber-attacks or regulatory issues as a response to human behavior and investors’ attitude towards this market. At the same time, the most debated factors explaining liquidity in the crypto-market are the price, trading volume,
 capitalization, fees, hash value (for PoW cryptocurrencies), and the size of the network \citep{Koutmos2018}.

It is important to mention the fact that liquidity is different from one cryptocurrency to another (the well-established ones are more liquid \citep{Wei2018, Koutmos2020}), as well as from one exchange platform to another. Despite the many benefits associated with liquidity, illiquid environments can also present some advantages, especially for the traders on this market, which can benefit from arbitrage opportunities and purchases at discounts \citep{Crowell2020}.


\underline{Likelihood:} Analyzed from the cryptocurrencies’ (crypto-assets that claim to be ‘money’) perspective, this risk would translate into an impossibility to be transformed in cash. That being said, one of the principal roles of money (being a medium of exchange) has just failed \citep{Greene2018}. There are many triggers behind crypto-assets illiquidity, among which: token supply algorithm, investors’ behavior, available supply, asset usage, fees, exchange
platforms failure, etc. As liquidity risk is already well-known in the financial markets (it is one of the determinants for market efficiency), we already know tools to measure it (trading volumes, book depth, and the
bid-ask spread, different liquidity ratios, etc.) \citep{Jain2018}. Similar to traditional securities, crypto-market
suffers from illiquidity during extreme price movement period of times \citep{Manahov2020}. A proof of market efficiency is the difficulty of manipulating prices. In the crypto-market, specifically concerning bitcoin, a significant herding behavior has been observed. The number of bitcoin whales increased to the impressive number of more than 2 thousand addresses \footnote{Owning between 1,000 to 10,000 BTC.} \citep{Bitcoin.com2020}. Besides the fact that herding implies a significant movement in prices (buy/sell large amounts of crypto-assets), it also has important supply implications as in the end, there are fewer assets available to trade \citep{Manahov2020}. 

Liquidity is an important characteristic of the market, influencing the investment costs and implicitly the desirability to trade. If we look at this risk from the bitcoin side, we could easily state that liquidity risk is very high. Moreover, \cite{Nguyen2019} shows that despite its market capitalization, bitcoin can be vulnerable to competition from new altcoins, as investors tend to diversify their portfolios and compensate for their decrease of bitcoin holdings with altcoins. On the other hand, if we look at the big picture, the one of crypto-market as a whole (not only bitcoin), where we have over 7000 crypto-assets available (coinmarketcap.com), we state that the likelihood is medium.  \\


\textbf{Supply risk} refers to the reserve available of crypto-assets. Some examples of important supply risk triggers are the loss of cryptographic keys (without which there is no possibility to access the afferent funds), cyber-attacks\footnote{e.g., the coins may stay blocked in the intruder’s account for a while, attempting to avoid the public eye. }, unclaimed rewards \citep{Coinmetrics.com2019}, reputation and the programmed limit of supplies. Not all cryptocurrencies have a maximum supply limit. For example, cryptocurrencies such as Bitcoin, Ripple, IOTA, Litecoin, and many others have a pre-established limited supply, while coins like Ethereum, Zcash, Monero, and others have no such limits. Following Rational Expectation Equilibrium models, the higher the supply uncertainty, the less informative crypto-assets prices will be. In this case, market prices are less efficient, and supply risk could thus even lead to an information risk \citep{Collomb2016}. Compared to fiat currencies, cryptocurrencies (especially bitcoin) were conceived as being less sensitive to market changes and inflation rate. However, with time we saw that Satoshi’s ‘perfect’ innovation leaves room for further improvement.


Mainly associated with market inefficiency at users’ and exchange platforms’ cost, the supply risk is affecting the mining and transaction validation processes, as well. Miners are vital in a PoW Blockchain performing both transaction validation and coin ‘minting’ functions. For successful work, they are rewarded by the system with an amount of newly created crypto coins. The reward offered by the system represents a method to create new coins and to increase the available supply of cryptocurrencies. At the same time, rewards are programmed to decrease steadily until the maximum supply is reached \citep{Eyal2018}. When this happens, the mining reward will be based only on transactions fees \citep{CryptoLi.st2020}.


Keeping in mind the above arguments, we state that the difficulty in creating (mine) new cryptocurrency, the supply limits, and the expenses incurred during this process, all significantly impact the supply imbalances and the final value of the assets.


\underline{Likelihood:} Since market liquidity is driven by the total supply available for trade, we understand that it is an important characteristic for market efficiency as well. Among the most notable triggers for supply issues, we have: token supply algorithm, hoarding behavior, loss of keys, wallet attacks, etc. \citep{Coinmetrics.com2019}. If the supply limits are not a risk for all the crypto-assets, it represents a threat at the market level concerning the leader bitcoin. As initially programmed, bitcoin’s maximum supply is 21 million coins. The already issued coins attain the approximate number of 18 million, supposing that the limit will be reached sometime around 2140 \citep{Ciaian2015}. As we already discussed the negative sides of limited supply (illiquidity and market inefficiency), we will now mention the bright side of this risk. Similar to commodities such as precious metals and natural gas, crypto-assets with limited supply attain high preference (subsequently high value), being regarded as ‘scare’ assets. By just looking at the price and market share of bitcoin, we can obviously observe that the
investor’s choices show a specific preference for this coin. In this case, the financial behavior within this market is under the influence of ‘scarcity gives value’ idea \citep{Verhallen1982}. However, this idea of value can bring important investment costs, as investors putting their money into such assets will consider asking for scarcity premiums on top of the existing ones for other risks \citep{Haase2013}. By assessing the supply risk at crypto-market
level, we state that the likelihood is medium.\\


\textbf{Environmental risk.} Known as an energy-gourmet, Blockchain
technology represents one of the key players in the fight towards the
green transition \citep{Charles2019}. This type of risk concerns
specifically the PoW Blockchains, which through their design, require
high computational power and much electricity for functioning
purposes. According to recent surveys, the bitcoin network is
responsible for using about 0.2\% of the global electricity and
emitting as much carbon dioxide emission as the country of Jordan
\citep{Irfan2019}. Another important aspect to mention is the
increasing number of ICOs, which require Ethereum Blockchain (PoW
based) for their smart contract application. According to the current
statistics, there are over three hundred thousand ether derived
crypto-assets (both active and non-active\footnote{tokens from former
  ICOs.} tokens) \citep{CryptoSlate.com2020}. We believe that the
technological constraints regarding electricity consumption should
receive priority consideration; perhaps very soon, the success of ICO
projects and the performance of businesses (using Blockchain
technology) will be influenced by environmental considerations. In the light of the current environmental context, there were many attempts to reduce the costs and unnecessary pollution,
although no significant progress was made so far \citep{Lasla2020,
  Saleh2021, Bentov2016, Lepore2020}. The emergence of mining pools,
the use of renewable energy (74\% of the used electricity is
renewable) and the lightning network, the emergence of platforms for
renting mining power (e.g., Nicehash) are the first steps towards a greener crypto world. We know anyway that there is a long road until we
reach the point of zero-emission power \citep{Irfan2019}. A solution
to stimulate a rapid transition to eco-friendly Blockchains could be
the implementation of a tax regime relative to the amount of energy
consumed or to the units of carbon emitted per transaction. In this
way, the crypto industry could become more aware of its environmental
impact, contribute to the domestic economy and hopefully, make an
effort to find the best alternative for both the ecosystem and
business \citep{Mecca2019, Goodkind2020}. Simultaneously, with the
increasing sensitivity of investors to the social responsibility of their
investment \citep{Brown-Liburd2015}, the assets showing negative
environmental externalities may be submitted to boycott from
investors. The environmental risk thus translates into a financial
risk.


\underline{Likelihood:} We know that during specific economic
conditions (pandemics, financial crisis, war, etc.), the stability of
financial markets can be highly affected. At the same time, as we
learn from the past events, such as the 2008 financial crisis or COVID
pandemics, the most performant and least risky investments, were the
socially responsible ones \citep{Lins2017, Singh2020,
  Palma-Ruiz2020}. Well-informed market players have concerns
regarding the enterprise risk management, financial performance and
considerations for the surrounding environments \citep{Ballou2006}. As
a strategy to decrease the risk exposure and make safer ‘investment
bets’, investors pay careful attention to what kind of assets they put
money in and make more socially responsible investments.


Once with the creation of crypto-derivatives and tokenized securities, we can consider that the first step towards convergence between the crypto world and traditional markets was done. Crypto derivatives can now be traded on both exchange platforms and OTC market \citep{DeribitInsights2020}. Brokers can switch from securities to crypto-assets or trade both. Regarding investment preferences, it was noticed that during turbulent periods and for safety considerations, investors tend to choose financial markets in favor of crypto-market \citep{Matkovskyy2019}. Taking into account the investors’ preference for ‘safety bets’ and concerns about environmental and social implications, it is believed that a more ecologically oriented Blockchain could significantly change the overall ‘safety’ perception \citep{Lai2021}. If this kind of risk does not have direct financial losses, it impacts the investment profitability, increasing the costs\footnote{E.g. A company issuing ICO projects can be directly affected by the investors’ social considerations, which will reflect in the amount of funds raised or the price/value of their crypto-assets (lower)} for financing. As time passes, investors give more attention to the crypto-market; therefore, we consider that for the moment, the likelihood is Medium. At the same time, we would like to mention that there are many chances that the likelihood becomes high if, from a technological point of view, nothing changes. \\


A summary of all financial risks discussed above will be presented in Table \ref{fin}.


\begin{table}[h!]
\centering
\caption{\label{fin}\textbf{Summary of financial risks}}
\setlength{\tabcolsep}{1pt}
\begin{tabulary}{\linewidth}{|c|L|L|c|}
\hline
\rowcolor[HTML]{FFFFFF}
\textbf{Risk}                                                                              & \textbf{Trigger}                                                                       & \textbf{Influence /  Consequences}                                                                        & \textbf{Likelihood}                              \\
\hline
\rowcolor[HTML]{EFEFEF}
\multicolumn{1}{|c|}{\cellcolor[HTML]{EFEFEF}}                                             & \textit{Cyber-attacks}                                                                 & \textit{•          Large loses for investors.}                                                              & \cellcolor[HTML]{EFEFEF}                         \\
\rowcolor[HTML]{EFEFEF}
\multicolumn{1}{|c|}{\cellcolor[HTML]{EFEFEF}}                                             & \textit{Technological risks}                                                           & \textit{•          A sign that the market is not stable and mature}                                       & \cellcolor[HTML]{EFEFEF}                         \\
\rowcolor[HTML]{EFEFEF}
\multicolumn{1}{|c|}{\cellcolor[HTML]{EFEFEF}}                                             & \textit{Regulatory mismatches}                                                         & \textit{•          Crypto assets trade with a risk premium relative to the risk investors may incur}           & \cellcolor[HTML]{EFEFEF}                         \\
\rowcolor[HTML]{EFEFEF}
\multicolumn{1}{|c|}{\cellcolor[HTML]{EFEFEF}}                                             & \textit{Human behavior}                                                                & \textit{}                                                                                                 & \cellcolor[HTML]{EFEFEF}                         \\
\rowcolor[HTML]{EFEFEF}
\multicolumn{1}{|c|}{\multirow{-5}{*}{\cellcolor[HTML]{EFEFEF}\textbf{Total market risk}}} & \textit{Reputation}                                                                    & \textit{}                                                                                                 & \multirow{-5}{*}{\cellcolor[HTML]{EFEFEF}High}   \\
\rowcolor[HTML]{FFFFFF}
\cellcolor[HTML]{FFFFFF}                                                                   & \textit{Lack of available information (e.g. white / yellow papers, inconsistent data)} & \textit{•          Financial loses for uninformed investors.}                                             & \cellcolor[HTML]{FFFFFF}                         \\
\rowcolor[HTML]{FFFFFF}
\cellcolor[HTML]{FFFFFF}                                                                   & \textit{Lack of knowledge/ understanding}                                              & \textit{•          Assets trade at prices far from their fundamental value}                               & \cellcolor[HTML]{FFFFFF}                         \\
\rowcolor[HTML]{FFFFFF}
\multirow{-3}{*}{\cellcolor[HTML]{FFFFFF}\textbf{Information risk}}                        & \textit{Reputation}                                                                    & \textit{}                                                                                                 & \multirow{-3}{*}{\cellcolor[HTML]{FFFFFF}High}   \\
\rowcolor[HTML]{EFEFEF}
\cellcolor[HTML]{EFEFEF}                                                                   & \textit{Regulatory mismatches}                                                         & \textit{•          Less investors}                                                                        & \cellcolor[HTML]{EFEFEF}                         \\
\rowcolor[HTML]{EFEFEF}
\multirow{-2}{*}{\cellcolor[HTML]{EFEFEF}\textbf{Liquidity risk}}                          & \textit{Reputation}                                                                    & \textit{•          Less efficient market}                                                                 & \multirow{-2}{*}{\cellcolor[HTML]{EFEFEF}Medium} \\
\rowcolor[HTML]{FFFFFF}
\cellcolor[HTML]{FFFFFF}                                                                   & \textit{Technological issue (supply limits)}                                           & \textit{•          Deflation, which can be a problem if crypto-assets will work as a method of payment}   & \cellcolor[HTML]{FFFFFF}                         \\
\rowcolor[HTML]{FFFFFF}
\cellcolor[HTML]{FFFFFF}                                                                   & \textit{Cyber- attacks}                                                                & \textit{•          Less efficient market}                                                                 & \cellcolor[HTML]{FFFFFF}                         \\
\rowcolor[HTML]{FFFFFF}
\multirow{-3}{*}{\cellcolor[HTML]{FFFFFF}\textbf{Supply risk}}                             & \textit{Loss of cryptographic keys}                                                    & \textit{}                                                                                                 & \multirow{-3}{*}{\cellcolor[HTML]{FFFFFF}Medium} \\
\rowcolor[HTML]{EFEFEF}
\cellcolor[HTML]{EFEFEF}                                                                   & \textit{Technological issue (PoW)}                                                     & \textit{•          Damage for the environment}                                                            & \cellcolor[HTML]{EFEFEF}                         \\
\rowcolor[HTML]{EFEFEF}
\cellcolor[HTML]{EFEFEF}                                                                   & \textit{Reputation}                                                                    & \textit{•          Crypto assets trade with a risk premium relative to their environmental externalities} & \cellcolor[HTML]{EFEFEF}                         \\
\rowcolor[HTML]{EFEFEF}
\multirow{-3}{*}{\cellcolor[HTML]{EFEFEF}\textbf{Environmental risk}}                      & \textit{Lack of regulation}                                                            & \textit{}                                                                                                 & \multirow{-3}{*}{\cellcolor[HTML]{EFEFEF}Medium}\\
\hline
\end{tabulary}
\end{table}



\section{Data analysis}

In line with the literature survey done in the previous section, here we are going to provide an example of how financial risk is linked to technological vulnerabilities. More specifically, we are going to assess if bitcoin’s volatility is affected by the events targeting the crypto-market. Some preliminary work on this problem has already been done by \cite{Corbet2020, Caporale2021, Grobys2021}.  \cite{Corbet2020} proved that (17) hackings that took place between 2017 and 2018 had affected the volatility and cross-correlation for the top 8 cryptocurrencies, while \cite{Grobys2021} showed how the (29) cyberattacks performed on bitcoin during the 2013–2017 period affected BTC and ETH returns.  \cite{Caporale2021} demonstrated how (4693) cyberattacks targetting not only the crypto-market and happening between 2015 to 2020 created spillovers and contagion effects among the top three cryptocurrencies. \\
In our analysis, we are using a sample of (53) events, which cover both the early times of bitcoin (2011-2013) as well as the hype period in 2018. \cite{Corbet2019} showed that among many factors, news related to cyberattacks have an important impact on the price movement of cryptocurrencies. As of January 2022, the amounts lost during our events (2011-2018) correspond to a 39 billion Eur (945,066 BTC) monetary equivalent. Given the extent of the losses incurred, it would be interesting to investigate if they impact the market. Therefore, in addition to what was already shown, more specifically, that cyberattacks events impact the price of cryptocurrencies, we want to take it a little further and see if the market is sensitive to the amounts lost. \\
This data analysis is an illustration meant to complement our previously performed survey. In accordance with the literature and with the aim to answer our research question: \textit{‘Can financial risks be triggered by technological vulnerabilities\footnote{83\% of the events considered represent attacks, while the rest of 13\% are malevolent actions that were possible thanks to Blockchain’s unique features; more details are provided in the next part.} of Blockchain technology?’}, we establish the following hypotheses:\\


\noindent \textbf{H1:Bitcoin’s volatility is positively linked to the number of events targeting the crypto-market.} \\
\noindent \textbf{H2: Bitcoin’s volatility is positively linked to the amounts lost due to these events targeting the crypto-market.}
\\ 

Similar to \cite{Corbet2020, Aliu2020, Akyildirim2020} and others, we retrieved the bitcoin prices from the Thomson Reuters Eikon database, while the list of events targeting bitcoin has been taken from \cite{Biais2020}. In total, our dataset comprises 53 events (see table \ref{events}), and the historical price data spans from August 2011 to September 2021.\\
In order to verify whether technological events have an influence on the risk of cryptocurrencies, we investigate the relationship between bitcoin’s volatility and the attacks on bitcoin. We check for the relationship between the volatility and the number of events, as well as the relationship between volatility and the amounts (in terms of bitcoin) lost as a consequence of these events. 


In the following section, we are going to compute volatility using the standard deviation method. Our choice is justified by the scope of this analysis: to demonstrate that there is a relationship between bitcoin’s volatility and our events. It is important to mention that for all our computations, our variables have been aggregated on a monthly basis. Table \ref{tab:descriptiveStatistics} displays the descriptive statistics of all variables used.\\


\begin{table}[h]
	\centering
		{
		\caption{Descriptive Statistics of all variables}
	\label{tab:descriptiveStatistics}
		\begin{tabular}{lrrr}
			\toprule
			 & volatility & event losses & number of events \\
			\cmidrule[0.4pt]{1-4}
			Median & 0.254 & 4736.000 & 1.000  \\
			Mean & 0.341 & 46743.630 & 1.407  \\
			Std. Deviation & 0.271 & 146592.481 & 0.636  \\
			Skewness & 2.306 & 4.591 & 1.343  \\
			Kurtosis & 6.909 & 22.177 & 0.832  \\
			Minimum & 0.089 & 8.000 & 1.000  \\
			Maximum & 1.359 & 748808.000 & 3.000  \\
			\bottomrule
		\end{tabular}
			\\ \footnotesize{\textit{The table summarises the descriptive statistics of all variables for the sample period.}}
	}
	\end{table}

The rationale behind choosing these events as proof of technological vulnerability is the following: most of them are attacks that show the vulnerability of this technology. Some events may represent malevolent actions (e.g., FBI seizes darknet operations) that were accomplished thanks to the distinctive characteristics of this market and which eventually demonstrate the vulnerability/drawback of the crypto-market. By distinctive characteristics of this market, we mean:


\begin{itemize}
\item Cryptocurrencies represent a virtual currency; built on open-source software code, they exist and operate just in the online environment. This makes them the target of cyberattacks that try to exploit any possible vulnerability of this technology.
\item Cryptocurrencies’ users need cryptographic keys in order to
access their funds or to place transactions. These keys easily become the source of attacks when they are not kept safely or if the code is easy to break.
\item The identity protection (anonymity) offered by Blockchain
technology attracted many enthusiasts; however, this feature makes it
almost impossible to catch the hackers/thieves.
\item The insufficient regulation and incertitude around
cryptocurrency world made them the perfect tool for the black markets;
these ones are also the few places accepting cryptocurrencies as payment.
\item The complicated nature of this technology and Blockchain illiteracy.The lack of proper understanding of how this crypto-world works was exploited in many forms to trick the users and steal their coins. An example would be the many scams performed by early crypto-exchange platforms.
\item Blockchain’s transactions are immutable. That implies as well the fact that in case of an attack, it is impossible to reverse (fraudulent) transactions or to recuperate the stolen funds. This characteristic, together with the anonymity feature, may incite malevolent actors to execute their plans.
\end{itemize}


For our data analysis, we compute the monthly standard deviation of bitcoin’s returns as:

$$ \sigma = \sqrt{\frac{(R_i - \mu)^2}{n}}$$

Where R is the bitcoin’s returns, $\mu$ is the average return, and n is the number of days of the window considered. 

Accordingly, with our hypotheses, we perform two correlation tests using the Pearson test and Spearman’s rank correlation. We want to measure the relationship degree between volatility and the number of events, as well as the relationship between volatility and the amounts lost during these events. Pearson, also known as a parametric correlation test, is one of the most common methods used in assessing the degree of relationship between two linearly related variables \citep{Pearson1932}. Spearman rho (a non-parametric test) measures the degree of association between two variables \citep{Spearman1904}. Both tests confirmed that bitcoin’s volatility is correlated (uncorrelated) with the number of events (the amounts lost during these events). The results can be seen in the bellow table
\ref{correlation}.


\begin{table}[h!]
\centering
\caption{\label{correlation}\textbf{Correlation tests for \em{volatility versus the number of events and the amounts lost during these events}}}
\begin{tabular}{||c c c c||}
 \hline
\textbf{Test} & \textbf{p-value} & \textbf{Correlation estimates} & \textbf{Variable}  \\ [0.5ex]
 \hline
 \textbf{Pearson} & 0.02156 & 0.4402268  &  \textit{number of events }\\
\hline
  \textbf{Spearman} & 0.03387 & 0.4095827  & \textit{number of events }\\
  \hline
 \textbf{Pearson} & 0.6606 & 0.08853083   & \textit{amounts lost } \\
\hline
  \textbf{Spearman} & 0.3888 & 0.4095827   & \textit{amounts lost } \\
\hline
  \end{tabular}
\\ \footnotesize{\textit{By looking at the p-values (0.02 $\&$ 0.03) resulted from our tests for correlation with the number of events, we observe that the results obtained are less than the significance level alpha = 0.05. Meaning that the monthly volatility of bitcoin and the number of events targeting it are correlated. At the same time, the high p-values (0.6 $\&$ 0.3) that surpass the significance level alpha of 0.05, prove that there is no correlation between bitcoin's volatility and the amounts lost due to events.}}
\end{table}


Furthermore, we want to check the relationship between bitcoin's volatility and the amounts lost and number of events together. In order to make this check we perform the following linear regression: 
$$ \sigma_{t} = \alpha + \beta_1 * EVENT_{number} + \beta_2 * EVENT_{amount} + \epsilon_{t}  $$

Where $EVENT_{number}$ is is the variable representing the monthly volume of events targeting bitcoin and $EVENT_{amount}$ is the monthly amounts lost, in bitcoins, due to these events. 


Our regression checks if there is a relationship between the monthly volatility of bitcoin and the monthly number of events with their respective losses. For the number of events, we obtain a $\beta_1$ of 0.193 with a p-value equal to 0.026. Meanwhile, for the losses incurred during these events, we obtain a $\beta_2$ of -7.589e-8 and a p-value equal to 0.832. Therefore, we conclude that our sample data provided enough evidence to show a relationship between the monthly volatility of bitcoin and the number of events targeting it. Concurrently, the results prove that there is no relationship between bitcoin volatility values and the amounts lost due to events. Detailed results are shown in the appendix section, table \ref{reg1}.

Our analysis shows that the volatility of bitcoin is not influenced by the financial losses incurred but rather by the number of attacks or other malevolent events targeting this market. This result, while in line with the existing literature \citep{Corbet2020, Caporale2021, Grobys2021, An2021}, proves that participants from the crypto-market are more sensitive to the number of cyberattacks than to financial losses. A way to justify this would be to analyze the discrepancy between the users’ expectations versus reality. Blockchain technology was created to offer a more secure and transparent alternative to the existing payment tools. However, that does not make it immune to cyberattacks, nor an absolute secure tool. \cite{An2021} has confirmed that cyber risks are negatively associated with cryptocurrencies’ success, damaging its reputation and investors’ trust.\\ Our results have important implications for the regulators working on the crypto-market. In the well-known paper ’Investor Protection and Corporate Valuation’, the authors \cite{Porta2002} state that \textit{“legal protection of investors is an important determinant of the development of financial markets. Where laws are protective of outside investors and well-enforced, investors are willing to finance firms, and financial markets are both broader and more valuable.”}
Technological vulnerabilities could perhaps be perceived as less harmful if the investors from the crypto-market are better protected. At the same time, we can see that the development of this market depends not only on technological innovation but also on the legal system that supports it. 


\section{Conclusion}

The crypto-market emerged in 2008, together with the first cryptocurrency created, bitcoin. Since then, Blockchain technology has evolved, potentially disrupting many fields beyond finance. However, still in infancy compared to its promised future, the crypto market has to overcome its many challenges. We believe that understanding and analyzing the crypto-market vulnerabilities represent the first step in overcoming its challenges. \\
In this paper, we perform a literature survey focusing on the types of risks present in the crypto-market. Our focus is on the technological and financial risks of crypto-market and Blockchain technology. First, we show that these risks can be related and that during specific market conditions, they can become a trigger one for another. Second, we offer a way to determine the likelihood of triggering financial risks through technological vulnerabilities. Here, we also emphasize the role played by financial behavior, social responsibility, and Blockchain literacy in the stability of crypto-market. Furthermore, to complete this study, we perform a short data analysis, demonstrating that cryptocurrencies’ price stability can be disrupted by technological vulnerabilities characteristic of this market. More research is needed on this matter, however, with the little data available, we showed that the bitcoin’s volatility level is influenced by the number of events targeting it. This evidence shows the implication of cybersecurity risks and poor regulation in the crypto-market development. \\
Our results support the general discussion from the literature survey while at the same time answer to our initial research question: ‘Can technological vulnerabilities of Blockchain technology trigger financial risks?’.
The empirical illustration provided in this article cannot be fully considered as empirical proof. This is mostly due to the size of our data. Broadly speaking, information related to the crypto-market is spread all over the internet, making it complicated for data collection and research. Up to this point in time, there is no
official or centralized database with attacks performed in the crypto-market, but rather a collection of mini statistics. On account of this, our limitation is reducing the possibility to perform empirical studies and accurately assess certain risks.\\
Finally, we conclude this survey with some research directions in an attempt to bridge a part of the existent literature gaps:

\begin{enumerate}
   \item There is a need for more research to increase Blockchain literacy. In spite of the growing interest in the crypto-market, practitioners are still challenged to transfer the Blockchain concept to market-oriented applications. General confidence in this new technology is often shattered by the negative news, scams, or attacks targeting this market. With their special features and exponential price changes, cryptocurrencies attract the attention of the large
public, including investors, researchers, regulators, or hackers. We believe that increased knowledge and understanding about these innovative technologies will better serve the participants within the crypto-market in making informed decisions; last but not least, it will help this market to evolve towards achieving its full potential. 
    \item Despite the growing number of empirical papers about crypto-market, we still lack the theory development in this field. With our study, we show that using the existing finance theories is insufficient if the technological characteristics of this market are not taken into consideration. Blockchain technology is not just a new tool; it represents a new way of doing business, a new operational system. Therefore, there is a need for more cross-disciplinary research that will take into account the important functions and implications of this technology (finance, regulation, cybersecurity, management, etc.).
     \item In recent years, there has been a growing awareness of climate change and environmental issues. Knowing that PoW cryptocurrencies represent a threat to our planet’s health, this subject needs more attention from both practitioners and academics. Investors represent an important group of stakeholders in the crypto-market. Before selecting their preferred investable assets, investors now pay more attention to their options and generally adopt
the ESG\footnote{Environmental, Social, and Governance conscientiousness.} evaluation criterion. With the ongoing pandemic and the continuous expansion of crypto-market, mainly based on PoW technology, we think that there is an urgent need for research addressing this challenge.
     \item In the course of the past decade, Blockchain has evolved while proving its capacity to disrupt various business sectors. Starting with an already complicated technology, namely cryptocurrencies, Blockchain development achieved high levels of both performance and complexity. Innovations such as ICO or DeFi\footnote{Decentralized finance} projects are built on stacks of complicated technologies, with each layer carrying an important amount
of (attack) risk. With that in mind, we argue that literature should address more the vulnerabilities and risks of this market, more specifically, the ones concerning other Blockchains than bitcoin. An assessment of the risks and vulnerabilities of the crypto-market as a whole could prevent investors from unnecessary losses, diminish the
number of low-quality products and increase performance and efficiency overall.
     \item As a decentralized system by design, Blockchain technology is not managed by any central authority but by its own algorithm, \textit{the code is law}. This leaves the duty of legal and international regulatory supervision in the hands of specialists from governments and industries. The only real progress in this direction started just in the beginning of 2017 \citep{BotoS2017}. Knowing that a large part of the vulnerabilities discussed in this survey would not have been possible if proper regulation was in place, we also consider this an area of further research.
  \end{enumerate}

We think this paper may be helpful for both academic researchers in their efforts to understand the determinants of the cryptoassets risk and to market participants (as well as cryptocurrency enthusiasts) for their investments.  \\


\pagebreak
\bibliography{library}
\bibliographystyle{apacite}
\newpage

\appendix
\section{Appendix}
\begin{table}[ht]
\centering
\caption{\label{reg1}\textbf{Linear regression 1}\\ \footnotesize{\textit{Summary of the OLS regression used to identify the relationship between monthly volatility and the number of events targeting bitcoin together with the losses incurred. Computations performed with R studio.}}}
\setlength{\tabcolsep}{1pt}
\begin{tabulary}{\linewidth}{llllll}
\hline
\textbf{Dep. Variable:}       & \multicolumn{2}{l}{Volatility}        & \textbf{Df Model:}                & \multicolumn{2}{l}{2}             \\
\textbf{Model:}               & \multicolumn{2}{l}{OLS}               & \textbf{R-squared:}               & \multicolumn{2}{l}{0.195}        \\
\textbf{Method:}              & \multicolumn{2}{l}{Least Squares}     & \textbf{Adj. R-squared:}          & \multicolumn{2}{l}{0.128}        \\
\textbf{Date:}                & \multicolumn{2}{l}{14 October 2021} & \textbf{F-statistic:}             & \multicolumn{2}{l}{2.913}          \\
\textbf{No. of Observations:} & \multicolumn{2}{l}{27}                & \textbf{Prob. (F-statistic):}     & \multicolumn{2}{l}{0.074}       \\
\textbf{Df Residuals:}        & \multicolumn{2}{l}{24}                & \textbf{Residual standard error:} & \multicolumn{2}{l}{0.253}        \\
\hline
\textbf{Coefficients:}   & \multicolumn{5}{l}{}                                                                                          \\
\textbf{Model}           &                  & \textbf{Estimate}    & \multicolumn{1}{c}{\textbf{Std. Error}}           & \textbf{t}        & \multicolumn{1}{c}{\textbf{p-value}}    \\
$H_{1}$          & $EVENT_{number}$ & 0.193             & \multicolumn{1}{c}{0.081}                       & 2.364             & \multicolumn{1}{c}{0.026*}  \\
$H_{2}$            & $EVENT_{amount}$    & -7.589e-8          & \multicolumn{1}{c}{3.534e-7}            & -0.215      &\multicolumn{1}{c}{ 0.832} \\
\hline
\end{tabulary}
\\
\footnotesize{\textit{With a \textbf{p-value} of 0.026*, a result that is significant and less than the significance level alpha: 0.05, we can conclude that there is a relationship between the monthly volatility and the number of events targeting bitcoin. At the same time, with a \textbf{p-value} of 0.832, a result that is higher than the significance level alpha: 0.05, we conclude that there is no relationship between the monthly volatility of bitcoin and the amounts lost during the events targeting it. }}
\end{table}


\begin{table}[ht]
\centering
\caption{\label{events} \textbf{Hacks, thefts and losses events related to Bitcoin} \\ \footnotesize{\textit{Source: \citep{Biais2020}}} }
\small
\setlength{\tabcolsep}{1pt}
\begin{tabular}{c|c|c}
\rowcolor[HTML]{EFEFEF}
\textbf{Date} & \textbf{Amount loss (BTC)} & \textbf{Description}                            \\ \hline
\rowcolor[HTML]{FFFFFF}
6/13/2011     & 25,000                     & Early user Allinvain was   hacked               \\
\rowcolor[HTML]{FFFFFF}
6/19/2011     & 2,000                      & MtGox theft - compromised   account             \\
\rowcolor[HTML]{FFFFFF}
6/25/2011     & 4,019                      & \textit{MyBitcoin theft   - wallet keys hacked} \\
\rowcolor[HTML]{FFFFFF}
7/26/2011     & 17,000                     & \textit{Bitomat loss - Wallet access   lost}    \\
\rowcolor[HTML]{FFFFFF}
7/29/2011 & 78,739 & \textit{MyBitcoin theft - wallet website   hacked}                                \\
\rowcolor[HTML]{FFFFFF}
10/6/2011     & 5,000                      & \textit{Bitcoin7 hack}                          \\
\rowcolor[HTML]{FFFFFF}
10/28/2011    & 2,609                      & \textit{MtGox loss due to hacking}              \\
\rowcolor[HTML]{FFFFFF}
3/1/2012      & 46,653                     & \textit{Linode hacks}                           \\
\rowcolor[HTML]{FFFFFF}
4/13/2012     & 3,171                      & \textit{Betcoin hack}                           \\
\rowcolor[HTML]{FFFFFF}
4/27/2012     & 20,000                     & \textit{Tony76 Silk Road scam}                  \\
\rowcolor[HTML]{FFFFFF}
5/11/2012     & 18,547                     & \textit{Bitcoinica hack}                        \\
\rowcolor[HTML]{FFFFFF}
7/4/2012      & 1,853                      & \textit{MtGox hack}                             \\
\rowcolor[HTML]{FFFFFF}
7/13/2012 & 40,000 & \textit{Bitcoinica theft - due to server   hack}                                  \\
\rowcolor[HTML]{FFFFFF}
7/17/2012     & 180,819                    & \textit{BST Ponzi scheme}                       \\
\rowcolor[HTML]{FFFFFF}
7/31/2012     & 4,500                      & \textit{BTC-e hack}                             \\
\rowcolor[HTML]{FFFFFF}
9/4/2012      & 24,086                     & \textit{Bitfloor theft - wallet keys   hacked}  \\
\rowcolor[HTML]{FFFFFF}
9/28/2012     & 9,222                      & \textit{User Cdecker hacked}                    \\
\rowcolor[HTML]{FFFFFF}
10/17/2012    & 3,500                      & \textit{Trojan horse}                           \\
\rowcolor[HTML]{FFFFFF}
12/21/2012    & 18,787                     & \textit{Bitmarket.eu hack}                      \\
\rowcolor[HTML]{FFFFFF}
5/10/2013     & 1,454                      & \textit{Vircurex hack}                          \\
\rowcolor[HTML]{FFFFFF}
6/10/2013     & 1,300                      & \textit{PicoStocks hack}                        \\
\rowcolor[HTML]{FFFFFF}
10/2/2013     & 29,655                     & \textit{FBI seizes Silk Road funds}             \\
\rowcolor[HTML]{FFFFFF}
10/25/2013    & 144,336                    & \textit{FBI seizes Silk Road funds}             \\
\rowcolor[HTML]{FFFFFF}
10/26/2013    & 22,000                     & \textit{GBL scam}                               \\
\rowcolor[HTML]{FFFFFF}
11/7/2013     & 4,100                      & \textit{Inputs.io hack}                         \\
\rowcolor[HTML]{FFFFFF}
11/12/2013    & 484                        & \textit{Bitcash.cz hack}                        \\
\rowcolor[HTML]{FFFFFF}
11/29/2013    & 5,400                      & \textit{Sheep Marketplace hacked \&   closes}   \\
\rowcolor[HTML]{FFFFFF}
11/29/2013    & 5,896                      & \textit{PicoStocks hack}                        \\
\rowcolor[HTML]{FFFFFF}
2/13/2014     & 4,400                      & \textit{Silk Road 2 hacked}                     \\
\rowcolor[HTML]{FFFFFF}
2/25/2014     & 744,408                    & \textit{MtGox collapse due to hacks   losses}   \\
\rowcolor[HTML]{FFFFFF}
3/4/2014      & 896                        & \textit{Flexcoin hack}                          \\
\rowcolor[HTML]{FFFFFF}
3/4/2014      & 97                         & \textit{Poloniex hack}                          \\
\rowcolor[HTML]{FFFFFF}
3/25/2014     & 950                        & \textit{CryptoRush hacked}                      \\
\rowcolor[HTML]{FFFFFF}
10/14/2014    & 3,894                      & \textit{Mintpal hack}                           \\
\rowcolor[HTML]{FFFFFF}
1/5/2015      & 18,886                     & \textit{Bitstamp hack}                          \\
\rowcolor[HTML]{FFFFFF}
1/28/2015     & 1,000                      & \textit{796Exchange hack}                       \\
\rowcolor[HTML]{FFFFFF}
2/15/2015     & 7,170                      & \textit{BTER hack}                              \\
\rowcolor[HTML]{FFFFFF}
2/17/2015     & 3,000                      & \textit{KipCoin hack}                           \\
\rowcolor[HTML]{FFFFFF}
5/22/2015     & 1,581                      & \textit{Bitfiniex hack}                         \\
\rowcolor[HTML]{FFFFFF}
9/15/2015 & 5,000  & \textit{BitPay phishing scam - hacker   takes over the CEO's accounts and access} \\
\rowcolor[HTML]{FFFFFF}
1/15/2016     & 11,325                     & \textit{Cryptsy hack}                           \\
\rowcolor[HTML]{FFFFFF}
4/7/2016      & 315                        & \textit{ShapeShift hack}                        \\
\rowcolor[HTML]{FFFFFF}
4/13/2016     & 154                        & \textit{ShapeShift hack}                        \\
\rowcolor[HTML]{FFFFFF}
5/14/2016     & 250                        & \textit{Gatecoin hack}                          \\
\rowcolor[HTML]{FFFFFF}
8/2/2016      & 119,756                    & \textit{Bitfinex hack}                          \\
\rowcolor[HTML]{FFFFFF}
10/13/2016    & 2,300                      & \textit{Bitcurex hack}                          \\
\rowcolor[HTML]{FFFFFF}
4/22/2017     & 3,816                      & \textit{Yapizon hack}                           \\
\rowcolor[HTML]{FFFFFF}
7/12/2017     & 1,942                      & \textit{AlphaBay admins assets sized by   FBI}  \\
\rowcolor[HTML]{FFFFFF}
7/20/2017     & 1,200                      & \textit{Hansas funds seized by Dutch   police}  \\
\rowcolor[HTML]{FFFFFF}
12/6/2017     & 4,736                      & \textit{NiceHash hacked}                        \\
\rowcolor[HTML]{FFFFFF}
6/20/2018     & 2,016                      & \textit{Bithumb hacked}                         \\
\rowcolor[HTML]{FFFFFF}
9/20/2018     & 5,966                      & \textit{Zaif hacked}                            \\
\rowcolor[HTML]{FFFFFF}
10/28/2018    & 8                          & \textit{MapleChange hack / scam}               \\
\hline
\end{tabular}
\end{table}



\end{document}
